\section{Key Management in Space DTNs}
\label{sec:survey}


The problem of key management is considered an open issue in DTNs \cite{menesidou2016automated}. This section focuses on the study of the proposed key management solution space DTNs.  The proposals were designed explicitly for space-based systems unless otherwise stated.



\subsection{Requirement for Key Management}

There are fundamental goals to any key management systems: secrecy of keys and assurance of purpose \cite{martineveryday}. DTNs present fundamental differences that make goals of protocols sometimes unclear or not simple to define. Routing is an example; in traditional networks, the purpose of routing is to select the best path between a source a destination. However, in space DTNs, connectivity is intermittent and propagation times are long. Hence the topology could change faster than topology updates propagation \cite{araniti2015contact}. Routing still consists in selecting the next hop until reaching a destination but the concept of best path is constrained.  The goal, in this case, could be to maximise the probability of bundle delivery.

Key management follows a similar problem. Many key management proposals evaluate the solutions using different metrics, and it seems that it is not clear the final requirements. For that reason, a new set of requirements is defined in this project. The new requirements complement the fundamental requirements mention before. These are classified into mandatory and optional. 

  
Most of the new requirements are proposed by Templin \cite{templin-dtnskmreq-00}. But not all requirements are considered. Nevertheless the ones omitted are discussed later. 



\begin{enumerate}
    \item Keys must be available when needed. The design must no relay on a query-response interaction between a source node and the trusted party or a destination node.
    \item The system must be trustworthy. There must be a trust anchor in the system; nodes cannot accept information directly from other nodes. 
    \item A single point of failure is not acceptable. Nodes cannot depend only on one entity, and some type of high availability must be present in the system.
    \item The system must support secure bootstrapping of nodes. The association between a node ID and its credentials must be certified before a node can use it in the DTN.
    \item The system must support delay tolerant key revocation. The system must be able to revoke credentials before the expiry date. 
    \item Credentials must be cached ``locally''. Space nodes must store credentials or the access to them should be subject to low delay and rarely disrupted links.
\end{enumerate}

HASTA ACA la correccion

The fundamental problem for DTN in space is that space nodes need authenticated information without online interaction with a CA, distribution centre or other nodes. Information refers to keys or some form of identification. The latter is employed in solution based on non-interactive protocols or Identity-Based-cryptography (IBC). Thus, the access to authenticated information is the main problem for DTN nodes and any key management system must solve.


\subsection{Analysis of current Key Management Solution}


The first question that arises is whether symmetric or asymmetric keys should be used to support a key management system.  The section  \ref{sec:space}concludes that symmetric keys are the most suitable solution for current space missions. However, for the Interplanetary Internet is expected a scenario similar to the terrestrial Internet \cite{rationale2010requirements} with multi-hop communications and cross organisational domains. Furthermore, policies, rights, hierarchy, and administrative groups should are part of the overall solution. Secure policies control resources usage, and bundle source and rights need to be identified. The network might scale to thousands of nodes, and different space agencies will participate in the network. Under this assumptions seems more suitable a key management solution similar to the Internet, symmetric keys might not be even acceptable in a cross-organisational domain \cite{ivancic2009security}.

The research effort so far is based on asymmetric keys and DTN working groups stated the need for automated systems for publication of public keys and revocations \cite{templin-dtnskmps-00}. Although there are ciphersuites based on symmetric keys in the Bundle Protocol Security Specification \cite{ietf-dtn-bpsec-07}, there is a common consensus that asymmetric keys will be the preferred method for the Interplanetary Internet. 

In public key management, a node must have an authenticated copy of the destination public key to encrypt the message, and the destination nodes must have the source node public key the verify the integrity. Key distribution is not a sensitive operation because the information is not secret. The common techniques are \textbf{pushing} from the key owner to the relying party and \textbf{pulling} from a usually trusted directory \cite{martineveryday}.

Low delays allow pushing and pulling, both to be effective on the Internet. For instance, TLS uses the pushing approach to retrieve the destination public key. After obtaining the public key, a node interacts with a Certificate Authority (CA) to check the validity of certificates. Alternatives for checking the validity are certificate revocation list CRL or the Online Certificate Status Protocol OCSP, but these requires interactions with the trusted authority which might not be feasible in deep space. Here should be noted as well that TLS, one of the most popular cryptographic protocol on Internet, uses a handshake between nodes to retrieve the public key and negotiate some parameters, protocols like this discourages for DTNs \cite{fall2003delay}.

There are modifications of OCSP, but these are designed to alleviate the overhead on certificates receivers but transfer the cost to nodes pulling the certificates. In Secure/Multi-purpose Internet Mail Extensions (S/MIME) sender could encapsulate its certificate as meta-data but the receiver is expected to validate it. S/MIME remove the communication between nodes but still require on-demands interactions with a trusted authority.

There are other alternatives to public keys based certificates. Templin \cite{templin-dtnskmps-00} states that web of trust might be an option for some types of DTNs, but more research should be done. For space-based networks, a model without a trusted authority is not acceptable. 

The other alternative is public key management based on Identity Based Cryptography IBC or Hierarchical Identity Based Cryptography HIBS. This approach eliminates the need for public certificates but introduces other issues on key revocation and key escrow. These problems can be addressed but no without adding overhead which questioned the benefits of IBC for DTNs. Farrell \textit{et al}. stated that IBC does not solve the problem \cite{irtf-dtnrg-sec-overview-06}; however, Asokan \cite{asokan2007towards} argue that IBC indeed solves the problem and provides an improvement in confidentiality services but not for authentication. In any case, proposal based on IBC are considered in this work.      


\subsection{Taxonomy of key Management in DTN}

A key management solution for DTN in space has the same requirement as any key management system plus extra considerations for the hostile environment. This section focuses on understanding why the problem is different from terrestrial networks as the Internet, define the requirements, and classify the solutions. 



Key management includes a wide range of processes which together provide security to cryptographic keys.  The problem of secure administration of cryptographic keys is complex but well-understood on terrestrial networks. Future space-based networks present two challenges to key management systems. Firstly, the network topology is dynamic; it consists of heterogeneous space nodes and planned communication links, opportunistic links might be available. Secondly, traditional cryptographic protocols are not suitable for this type of network, as mention in section \ref{sec:dtn}.

Menesidou \textit{et al.} \cite{menesidou2017cryptographic} classify key management schemes into three categories depending on whether the solution deal or not with secure initialization, key establishment, and key update/revocation. Key establishment is divided again into two groups: two-party and group communication. The same approach is adopted in this report. As this project aims to choose the most suitable key management scheme for the Interplanetary Internet, assumptions and benefits are scrutinised. The analysis considers proposals which focus only on key establishment because this is the fundamental problem for DTN in space.

\subsection{Standardisation effort}

Although the DTN working groups have been active since 2007, a key management specification has been postponed for many years, mainly for its complexity \cite{rfc6257}. Farrell present a DTN Key Management Requirements \cite{farrell-dtnrg-km-00}. However, this document is a high level document and does no establish concluding requirements for a key management solution. 


The first advance was a problem statement by Templin in 2014  \cite{templin-dtnskmps-00}. This internet-draft assumed that the solution involve some kind of public key cryptography and an automated system for publication of certificates and revocation lists. More important, the system must be designed to continue operation in the presence of long delays and intermittent connectivity and traditional key management protocols do not satisfy the requirements. The author suggest as well, the use of one-time keys, so each bundle is encrypted with a symmetric key encrypted by the destination public key. 

Another Internet-Draft more detailed was published by the same author one year later \cite{templin-dtnskmreq-00}. The document proposed nine requirements for a key management solution. The requirement include no single point of failure, multuopke points of authoprity and delay tolerant key revocation. The document also proposed four design criteria to match ensure that the solution match the requirements. Among these stand out the publisher/suscriber model instead of an online directory service and the multiple source of publications. 

Viswanathan \cite{viswanathan-dtn-pkdn-00} enumerates the possible communication patterns, data structure, architectural elements, and trust relationships.  This Internet-Draft provide a high level solution for key distribution and revocations using the publisher/suscriber pattern following the approach of previous documents.

Finally, a newly published document proposed the Delay-Tolerant Key Administration (DTKA) which is a system for key management intended for use in DTNs, particulary space-based communications. 



\subsection{A Survey of Key Management in space DTN}

In the beginning, the research effort focused on modify public key management based on certificates and IBC solutions. 





