
The operations \verb|\left| and \verb|\right| are useful for making
braces the right size:
\[
   \left\{ 0, \frac{1}{2}, 1 \right\}, \;
   \left( \sum_{i=1}^3 (i^2 + 2) \right), \;
   \left[ 1 + \frac{1}{2 + \frac{2}{4 + \frac{1}{5}}} \right].
\]

Here is a table:

\vskip 0.2cm

\begin{center}
\begin{tabular}{|l|l|}
  \hline
  $N$ & Information about $N$ \\
 \hline
  2 & A prime \\
  3 & A prime \\
  4 & A square \\
  5 & A prime \\
  6 & Half a dozen \\
  \hline
\end{tabular}
\end{center}

\vskip 0.2cm

In the next section you will find Theorem \ref{main-thm}.

If you want to start on a new page then do this:

\newpage

\section{A theorem}

\begin{theorem}
\label{main-thm}
Let $E/F$ be an elliptic curve defined over a number field $F$.
Let $\End(E) = \OO$ be an order of discriminant $D$.
Let $p$ be a prime for which $E$ has good and supersingular reduction.
Let $\wp $ be a prime ideal of $F$ above $p$. Let $\tilde E$ 
over $k=\F_{p^m}$ be the reduction mod $\wp $ of $E$.
Let $\pi$ be the $p^m$-Frobenius map on $\tilde{E}$.
Suppose $r \mid \# \tilde{E}( \F_{p^m} )$ is a prime
such that $r > 3$ and $r \nmid pD$.

Let $d\in \N$ be such that $\sqrt{-d} \in \OO$.
Let $\Psi \in \End (E)$ satisfy  $\Psi^2=-d$.
Let $\psi \in \End_{\F_p}(\tilde E)$ be the 
reduction mod $\wp$ of $\Psi$.
Then $\psi$ is a suitable distortion map for points $P \in \tilde{E}[r]$
which lie in a $\pi$-eigenspace.
\end{theorem}


\begin{proof} 
You don't want to see the proof. \hfill $\Box$
\end{proof}


\section{More things}

\subsection{Subsections}
\label{sub-sec}

This is subsection \ref{sub-sec}.


\subsection{Spot the difference}


Experts in Latex find that they like things a certain way,
for example:
\begin{itemize}
\item ``quotes'' rather than "quotes".

\item $a \mid b$ and  $a \nmid b$ rather than
$a | b$ and $a \not| b$.

\end{itemize}


Doing references the right way is also important.
Some examples are given below. 


\begin{thebibliography}{}

\bibitem{boneh} D. Boneh,
The decision Diffie-Hellman problem,
in J. Buhler (ed.), ANTS III, Springer LNCS 1423
(1998) 48--63.

\bibitem{cohen} H. Cohen,
{\it A course in computational algebraic number theory},
Springer GTM 138 (1993).

\bibitem{gross} B. H. Gross,
Heights and special values of $L$-series, 
CMS proceedings, \textbf{7}, AMS (1986), 115--187.

\bibitem{velu} J. V\'elu,
Isog{\'e}nies entre courbes elliptiques,
C. R. Acad.\ Sci.\ Paris,
S{\'e}rie A, 273 (1971) 238--241.

\end{thebibliography}

\end{document}