\section{Conclusions}
\label{sec:con}

The first days of space exploration required relatively simple communication systems, but future space missions envision a shift from point-to-point links towards a multi-hop communication, and from an isolated environment to a cooperative model between space agencies.

Traditional protocols are not suitable for interplanetary communication, the presence of celestial bodies affects the availability of links and considerably distances result in long propagation delays. The Delay Tolerant Architecture and the Bundle Protocol define an overlay layer which provides a solution to these problems.

Security is a critical component in space system as resources are scarce and expensive, especially in the space segment. The Bundle Security Protocol defines integrity and confidentiality services for the Bundle Protocol.  However, this security specification assumes that bundle nodes are in possession of cryptographic keys. 


Key management is recognised as one of the most challenging problems in Delay Tolerant Networks, and there is no common consensus on which solution suit best to future interplanetary networks. Bulk encryption and symmetric keys are commonly used to secure communication in space, but his model does not fit with the new requirements, and an "Internet-style" model is expected for key management. This work presented an analysis of key management systems for the envisioned Interplanetary Internet. 

%This work classifies key management systems depending on whether the scheme deals with secure initialisation, key establishment and revocation, which are the most challenging parts of the life-cycle.  

%\subsection{Conclusions}

%The conclusions of this work are listed below


%It is well-known that the problems of public key management lie in the deployment and administration of the infrastructure to support cryptographic keys rather than cryptographic protocols. The first challenge was to understand why this problem is different from terrestrial networks and define the requirements for a suitable solution. 

First, this work identifies the access to authenticated information without online interaction with any party as the fundamental difference between traditional and delay tolerant key management. Network partition is a typical situation in space DTNs, and the connection to a directory server or the other party cannot be assumed. 

%Moreover, the communication paths between  (sender,receiver), (sender, trusted party), and (receiver, trusted party) are not uniform, resulting in variable delays. 


Second, this work defines seven requirements to evaluate state-of-the-art key management solutions. None of the evaluated solutions satisfies all requirements, but the Delay-Tolerant Key Administration system  (an IETF Internet-Draft) achieves six of them, leaving scalability as an issue to be improved. 
%These requirements encompass the availability of keys, trust anchor, availability of the system, secure initialisation, revocation, scalability and compatibility with Bundle Protocol.

The professionals from the space industry anticipate that no single interplanetary network will be deployed in Solar System. A bunch of disjoint application domains administrated by different groups will be the most likely scenario to the disadvantage of all participants.

Each application domain will have different infrastructure and requirements; in consequence, no single key management system fits all application domains. The stakeholders will define which authority model will be used for each interplanetary network.  

The clear separation of protocols, policies, ciphersuites, and configuration will make the operation of networks simpler. Hierarchical implementation will suit better different application domain requirements.

Geopolitical aspects affect the future deployment of interplanetary networks. Currently, there is no cooperation between nation states such as China and the United States. However, space exploration has been a place to bring nations closer, and this can happen again with the help of other organisation such as the European Space Agency.


There is still work to do on key management for Interplanetary networks. Remain as future work the scrutinising of the Delay Tolerant Key Administration system. Scalability and the concept of in-space directory services could be further explored. The definition of policies and configuration for interplanetary networks remain pending, as well as its standardisation. The developing of application for DTNs present a change of paradigm which remains without much attention.

\subsection{A Personal Voyage}

I decided to study the problem of key management for the Interplanetary Internet because it encompasses two of my favourites topics: space exploration and protocols.  I will explain the journey over this three months, the challenges, limitation, errors and more important, lessons learned.


I had several challenges to solve before analyse key management systems. First, I had to understand how space missions communicate with spacecraft, and second I had to find a way to contact experts in the space industry. 

The first problem involved reading materials about architecture, communications protocols, and security for space missions. It was a big challenge that required a long time. I focused on the cryptographic and network topology aspects of key management, but aspects as compliance restrictions, process and environmental control were barely considered.


A good experience was the contact with professionals from the space industry. People from NASA and Surrey Satellite Technology SST were keen to give their feedback on many issues. A visit to SST facilities in Guildford helped me to understand the problem from the mission operation centre and security issues. 

The project had some unexpected situations. First, the idea was to analyse state-of-the-art solutions because there was no proposed solution by the IETF. In the middle of the project, I found a newly published Internet-Draft containing a full specification for a key management system for space DTNs. Of course, this is an Internet-Draft,  the DTN working group have not analysed yet, but the scenario changed, and I had to adapt my project. 

The second situation was the feedback from the people from NASA. They expect many interplanetary networks and different key management system for each one. This contrasted with my idea to find a key management system that suits best to the Interplanetary Internet. Again, I had to adapt my project and explore new parts of the problem such as the geopolitics issues between the stakeholders.

My most important learned lessons are: 1) A project can change direction quickly, and one has adapt to the circumstances. 2) Implementations constrain does limit the scope of research, and even if it is not likely a unified interplanetary network, this does not imply that a theoretical solution is not possible. 

 
%Lower operational cost, highest data return and more floxible system that can support the growthing number of spacecraft are other motivations. 

 

%``The general concern of DTN is How to address the architectural and protocol design principles arising from the need to provide interoperable communications with and among extreme and performance-challenged environments, where continuous end-to-end connectivity cannot be assumed'' \cite{caini2011delay}.

%A space DTN is a particular case of delay tolerant network. By far, is the most extreme scenario where delays could be extremely long and the communications get disrupted by the planets orbit and motion.





% This work has been about building 

% Significant work has been done
% It is important to consider 
% The most significant limitation to the 
% As has already been made clear
% It is also important to investigate 
