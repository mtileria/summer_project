\documentclass[11pt]{article} % do not change this line
\input{BigDataStyle.txt}      % do not change this line

\usepackage{amsmath,amsfonts,amssymb,amsthm,latexsym,graphicx,caption,hyperref,listings,color,xcolor}
\usepackage[export]{adjustbox}
\usepackage{float} 
\usepackage[section]{placeins}
\usepackage[boxed,ruled,vlined,dotocloa,english,onelanguage]{algorithm2e}
\graphicspath{ {pictures/} }
\emergencystretch=5mm
\tolerance=400
\allowdisplaybreaks[4]


\theoremstyle{plain}
\newtheorem{theorem}{Theorem}[section]
\newtheorem{proposition}[theorem]{Proposition}
\newtheorem{corollary}[theorem]{Corollary}
\newtheorem{lemma}[theorem]{Lemma}
\newtheorem{problem}[theorem]{Problem}

\theoremstyle{definition}
\newtheorem*{remark}{Remark}

\title{Key Management for the Interplanetary Internet}
\author{Marcos Tileria}

\newcommand{\Programme}{Doctoral Training Centre in Cyber Security}


\begin{document}
\maketitle

\declaration

\begin{abstract}
  %The Interplanetary Internet is a conceived computer network designed to provide Internet-like services to space missions. Communications in deep space is characterised by extremely long delays and the lack of end-to-end connectivity. Communication infrastructure in deep space is scarce and unauthorised access to the them is a critical issue. In this context, the Delay/Disruption Tolerant (DTN) Architecture was defined to provide a reliable communication system for ``challenged networks''. Cryptography key management is considered the most difficult security problem for DTNs, and there is no definitive solution to this problem. 
  
 The Interplanetary Internet is a conceived computer network designed to provide Internet-like services to space missions. Communication in deep space is characterised by extremely long delays and the lack of end-to-end connectivity. Communication infrastructure in deep space is scarce, and unauthorised access to them is a major problem. Cryptography key management is considered one of the most difficult security problems for the Interplanetary Internet, and there is no widely accepted solution to this problem. 
  
\end{abstract}

\section{Introduction}


Different research groups
experiments
why no tcp and others
other protocols: LTP, CFDP, 
security

Cryptographic Key Management is recognised as the most difficult problem in DTNs. For many years, research and working groups from IETF and other organisation left the problem unsolved and assumed that there was a reasonable key management solution that can be used to generate, update and distribute keys among DTN nodes. 

Key management is a central part of any secure communication system and plays a critical role in overall security. Furthermore, key management is recognised as a difficult problem in a connected network, and a disrupted network subject to long delays make the problem even more complicated.  


The future space exploration envisions multiple space nodes that communicate with each other and not only between space and Operation Centre (OC). A space node could be any human-made object send into deep space in support of space mission, for instance, satellites, orbiters, landers, rovers or any small robot. An immediate consequence of the new communication model will be an increase in the number of interconnected systems, data transfer and connectivity. The current architecture is not designed to support such expansion, and an upgrade will be necessary. A similar situation occurred in the early days of telephone systems and switchboards. As the number of users grew, it was not possible human circuit switching anymore, and the system had to evolve into an automated switching systems \cite{rationale2010requirements}. The space communication architecture requires a transformation from point-to-point communications toward a flexible architecture capable of managing the new communication requirements of future space missions. 


 \newpage

\section{DTN Space Networks}


In \cite{cerf2007delay}, the authors define the Interplanetary Internet as ``a communication system design to provide Internet-like services across Interplanetary distances in support of deep space exploration''. The original motivation of the Delay Tolerant Architecture was the design of the Interplanetary Internet Architecture. Then, the specification evolves to cover a wide range of delay tolerant environments like wireless sensor networks, satellites networks, military tactical networks and others, and end with the specification of the Bundle protocol \cite{scott2007bundle} and a series of related documents.  

There are fundamental differences in the way communication is performed in the space. Some of these properties make well-known protocols for terrestrial networks unsuitable for space networks \cite{fall2003delay}. The most significant differences between communications in the space and terrestrial communication are the extremely long signal propagation due to the speed of light and the lack of end-to-end connectivity due to planets motion which results in the disruption of connectivity between the OC and the spacecraft. However, there many other problems related to communication in deep space. Low data rates and highly asymmetric links, typically 8-256 kb/s for up-links (telecommands) and a much higher down-link data rate for telemetry commands and science data, in order of Mb/s, but for extremely long distance as Mars, the data rate is between 128 kb/s and 256 kb/s using relay orbiters. A centrally managed access to communications channel reduce the likelihood of congestion scenarios but generates intermittent scheduled connectivity limited by mission policies \cite{burleigh2003delay}.

Nowadays, a limitation in space missions is that the predominant communication model is point-to-point between a spacecraft and its mission control centre and each mission operates independently. This communication model have been implemented using bespoke communication protocols for a number of years. However, over the last years, there is an effort from CCSDS and other bodies like IETF to standardised communication protocols for space missions. For instance, CCSDS developed the Space Packet Protocol specification and is working on its particular definition of the Bundle protocol for space communication \cite{standard2010ccsds}. 

The disruption of communication in Mars due to celestial mechanics and the low data rate motivated the use of a hop relaying scheme to send science data but without implementing true inter-networking. Under this configuration, Mars rovers use available orbiters as intermediate hops but there is no addressing scheme, no classes of data and no proper network layer. These limitations will restrict operations of future missions which will gather and send more science data to mission control centre  \cite{rationale2010requirements}. 

The experience has shown the advantages of a multi-hop communication model over a point-to-point model. Some of them are: an increase in science data return, lower power and hardware requirements for constrained nodes like rovers,  and the most important benefit is more contact opportunities \cite{rationale2010requirements}. Besides, it is predicted that future space mission will operate in an environment of interoperability and cross-support across space agencies. Spacecraft, satellites, rovers and other human-made objects will act as a network of space-based entities as shown in figure XXX.

INSERT FIGURE!!!


\subsection{Security in Space Communications}



Today's ubiquitous cyber threats make space missions target of malicious attackers.  In the past, only military missions were targets of adversary attacks and for that reason have been highly protected, but this is not true anymore, and all mission requires protection. The advance of technology and communication systems apply to civilian and scientific missions in deep space. Furthermore, satellites form part of many critical systems like navigation, weather study, and disaster response activities.  As state before, future missions will require interoperability between nodes that might be administrated by different agencies. Therefore, a space mission must consider the implementation of secure communication to reduce the likelihood of attacks whether its infrastructure is in orbit around the earth or exploring remote places in the solar system \cite{book2006security}.

As the traditional communication model in space missions is point-to-point between operation centre and spacecraft, data links have been secured using bulk encryption. Although this technique is simple, requires special techniques and hardware to be deployed in both sides and it is not scalable for the Interplanetary Internet model which allows communication over multiple hops and interoperability among space agencies. The International Traffic in Arms Regulations might have jurisdiction for international interoperability \cite{ivancic2009security}. 

The CCSDS present a list of threat sources applicable to space missions. The adversaries include terrorist, criminals, foreign intelligence services, computer hackers, and commercial competitors. There are other sources as insiders, environmental and structural. The document classifies threats as active (jamming, unauthorised access, masquerading, Denial of Services) and passive (tapping, traffic analysis). 

It is worth to mention that threats could apply to any segment of space missions: space segment, ground segment, space-link segment. A simplified diagram of segments is shown below ....

INSERT FIGURE!!!!

In this example, the satellite and the lunar rover form part of the space segment. The space-link segment is the connection between the satellite and the ground station and the ground segment is composed of the ground station and the mission control centre which has the administration of the elements in space. Two important points should be noted in this diagram.  Firstly, the mission control centre and the ground station are not in the same physical location, which requires a secure connection between them. On the other hand, the network infrastructure on the earth could belong to one entity and the mission control centre to another. An example is the Mars Exploration Mission, which has its mission control is the JPL headquarters and uses the Deep Space Network as the ground station. This situation suggests that a key management solution for the Interplanetary Internet should consider interoperability not only between nodes from different agencies, but it should also consider interoperability between the Internet on earth and the key management solution for the space DTN. 

 In \cite{book2012architecture}, CCSDS present security requirements for five space mission profile. These profiles are human spaceflight, earth observation, communication, scientific, and navigation. For instance, humans spaceflights present all security requirements but also ``safety-of-life'' and privacy issues. The security of earth observation, navigation, and communications missions vary depending on the information value and the relative position to the earth. Some missions require security for telemetry, telecommands, and payload communication, others only need to secure a subset of those. Science missions are more complicated than the previous,  the distance to the earth change requirements dramatically. For interplanetary missions, there are extra factors that should be considered at the time of implementing security, for instance, communication delay, discontinuous communication, fault tolerance, ability to use intermediate nodes (planned and unplanned), significant mission lifetime. The last point to consider is multi-organisational missions in which payloads and data may belong to different organisations. Relay spacecraft and the origin or destination may be administrated by different agencies as well. 

It remains to see how different profiles could fit in a single key management scheme.
 



\subsection{Key Management for current Missions}

In current missions, key management complexity is low. There are mainly two entities involve: the mission control centre or operation control centre OCC and the spacecraft. The OCC is responsible for generation, managing and revoking keys. In the context of human-crewed missions, the human intervention in spacecraft is limited. Optionally, the ground station could take part in the key management in the case it is working as security gateway and user facilities may be present if payload data have to be disseminated \cite{book2011space}. 

Differences between current mission and future missions:

- key generation procedure affects only the ground segments. 

It is recommended to store keys in secure storage such as smart card and physical protected place. Master key is used to exchange traffic protection keys TPKs. this process may takes place inside the operational network. 

Initial pre-distribution of symmetric keys. Master keys are used to generate or exchange new keys of a lower hierarchy as a traffic protection key. In the case that a master key is used as encryption key, all key management operation are conducted before launch. Could be more than one master key \cite{book2011space}.

Many environmental properties influence a key management solution in space missions \cite{book2011space}. Asymmetric channels and bandwidth restriction, propagation delay, intermittent connectivity, remote location, limited computational resources and memory. These limitations imply that symmetric key cryptosystems are more suitable in space missions. 

However, for the Interplanetary Internet is expected a full constellation scenario in which each spacecraft, robot,  ground station, OCC, and end user facilities are individuals nodes in a global network using the Bundle Protocol. Under this conditions, it seems more likely a scheme similar to terrestrial Internet as not only keys are in consideration, policies, rights, hierarchy,  and administrative groups should be consider. Moreover, the network might scale to thousands of nodes, and symmetric keys deployment becomes more difficult considering that different organisations participate in the network.   Heterogeneous nodes will operate in the Interplanetary Internet, and some of them may have an active role in the key management scheme. All of this, support the idea of a similar Internet architecture for key management with some extension to handle particular constraints that present the deep space environment like disruption and long delays.  


from DTN space requirement: LLC o LLT


\subsection{Delay and Disrupted Tolerant Networks}

The delay Tolerant Architecture is proposed in \cite{cerf2007delay}. It defines a network architecture for irregularly connected networks subject to frequent partition and possibly long propagation delays. To face these particular properties, the authors define an overlay layer which provides end-to-end reliable messages delivery called the \textbf{bundle layer}. This layer works under the application layer and could be used over the transport, network or data link layers using different convergence layer for each case to communicate with lower layer protocols. 

The DTN architecture provides persistent storage, hop-by-hop transfer, late binding, optional end-to-end acknowledgement to offer reliable delivery of messages. It also uses URI identifiers for naming; this flexible model allows the encapsulation of different addressing schemes. The Bundle protocol specification \cite{rfc5050}, produced by the IETF,  define the services offered by the Bundle protocol, bundle format, bundle processing and convergence layers. A node implementing the bundle protocol is called bundle node. 

 




\subsection{Security in space DTN}


 From the beginning, the Delay-Tolerant Research Group worked on the security specifications for the Bundle protocol. The motivation was to provide data integrity and confidentiality services to the bundle layer. The Bundle Security Protocol Specification \cite{rfc6257} partially defines the security blocks, security processing, allowed ciphersuites, tunnel encapsulations, policies and key management. The Bundle Protocol security Specification \cite{ietf-dtn-bpsec-07} extends the previous document adding the design decisions, canonical form of blocks, security considerations, adversary model, and more.  There are many other documents related to security which complement these specifications.  It is worth to mention that many specifications are still \textit{work in progress} and there are issues that remain unsolved.
 
 A space DTN operates as an overlay network on the top of different lower layers. Thus, the first threat are non-DTN nodes which could exploit vulnerabilities in the bundle layer. Besides, the Bundle protocol allows a mix of security-aware nodes and non-secure nodes in the same network; in a space scenario, this could be for the lack of node resources or mission policies, for instance, security being implemented in other layers. Unauthorised access is a major problem for a space DTN because the resources are scarce, especially in the space segment. Denial of service (DoS) attack is another concern for a space DTN. Attackers could take advantage of long delays to make DoS attacks more effective. 
 
 
 
 




 \newpage

\section{Delay and Disrupted Tolerant Networks}
\label{sec:dtn}

The delay Tolerant Architecture \cite{cerf2007delay} defines a network architecture for irregularly connected networks subject to frequent partition and possibly long propagation delays. To face these particular properties, the authors define an overlay layer which provides end-to-end reliable messages delivery called the \textbf{bundle layer}. This layer works under the application layer and could be used over the transport, network or data link layers using different convergence layer to communicate with lower layer protocols. 

The DTN architecture provides persistent storage, hop-by-hop transfer, late binding, and optional end-to-end acknowledgement to overcome the constrained environment. It also uses Universal Resources Identifiers URI  as naming scheme; this flexible model allows the encapsulation of different addressing schemes and late binding. The Bundle protocol specification \cite{rfc5050}, produced by the IETF,  define the services offered by the Bundle protocol, bundle format, bundle processing and convergence layers. 

A bundle is the basic data unit of the Bundle protocol. The bundle is a self-contained data unit because negotiation between nodes might not be possible due to the delay and disruption. A bundle node could be any entity that can send or receive bundles. From here on, spacecraft, satellites, rovers or any human-made object in deep space can have one or more bundle nodes. 

A space DTN is a particular case of delay tolerant network. By far, is the most extreme scenario where delays could be extremely long and the communications get disrupted by the planets orbit and motion. There is much work ongoing on the Bundle protocol, and it is believed that in a few years the entire specification could be ready to deploy in space missions. As state before, several tests were already successfully performed of the Bundle protocol in deep space.




\subsection{Security for DTN in space}


 From the beginning, the Delay-Tolerant Research Group worked on the security specifications for the Bundle protocol. The motivation was to provide data integrity and confidentiality services to the bundle layer. As space missions are more connected to the Internet and the economic value of space assets are very high is comprehensible consider information security as a critical component of the communication system.
 
 The Bundle Security Protocol Specification \cite{rfc6257} partially defines the security blocks, security processing, allowed ciphersuites, tunnel encapsulations, and policies. The Bundle Protocol security Specification \cite{ietf-dtn-bpsec-07} extends the previous document adding the design decisions, canonical form of blocks, security considerations, adversary model, and more. There are many other documents related to security which complement these specifications.  It is worth to mention that many specifications are still \textit{work in progress}.
 

A space DTN operates as an overlay network on the top of lower layers. Thus, the first threat is non-DTN nodes which could exploit vulnerabilities in the bundle layer. Besides, the Bundle protocol allows a mix of ``security-aware'' nodes and ``non-security-aware'' nodes in the same network; in a space-based network, the reason to deploy non-security-aware nodes could be for the lack of physical resources or security being implemented in other layers \cite{rfc6257}. 

Unauthorised access is a major problem for a space DTN because the resources are scarce, especially in the space segment. Denial of service (DoS) attack is another concern for space DTNs. Attackers could take advantage of long delays to make DoS attacks more effective \cite{rfc6257}. 

 
\subsection{Key Management for DTN in space}

The last version of the Bundle Security Specification \cite{ietf-dtn-bpsec-07} defines two types of security blocks: Block Integrity Block (BIB) and Block Confidentiality Block (BCB).  An important requirement of services provided by these security blocks is that intermediate security-aware nodes may verify the integrity of the bundles or decrypt them. The consequence is that security-aware-nodes may need to be in possession of the corresponding verification or decryption keys. This requirement implies that any key management solution must consider this situation to meet the security specification. 


For the Interplanetary Internet is expected a full constellation scenario similar to the terrestrial Internet \cite{rationale2010requirements}. Under this assumption, seems more suitable a key management design similar to the terrestrial Internet. Furthermore, policies, rights, hierarchy,  and administrative groups should be considered as well.  The network might scale to thousands of nodes, and different space agencies and private companies will participate in the network. Therefore a symmetric key solution becomes complicated.  

Symmetric keys may be a requirement even for future space missions. 
There are sensible operations which require symmetric keys between the spacecraft and the mission control centre. Examples include recovery operations, low-level telemetry and low-level telecommands.  However, there are some ideas on the IETF and Internet Society working groups involving the use of applications on the top of the bundle layer to handle these operations. 


There is always a trade-off of usability against security. Security adds overhead, but the limited resources of some DTN nodes in the space segment requires the minimisation bandwidth and storage overhead, otherwise, mission planners will reject the security implementation \cite{book2012architecture}.   

The next section is a literature review of the so far proposed key management solutions for delay tolerant space-based networks.












\newpage

\section{Key Management in Space DTNs}
\label{sec:survey}




The problem of key management is considered an open issue in DTNs. \cite{menesidou2016automated}.
This section focuses on the study of the proposed key management solution for space DTNs.  Some proposals were designed for space-based systems and others for general DTN.


%\subsection{Requirement for Key Management}


It is already complex to deploy key management systems in traditional networks. The problem lies in the deployment and management of the infrastructure to support authenticated keys and not cryptographic algorithms \cite{al2003certificateless}. For DTNs, the problem is even worst.


A Key management system has fundamental goals: secrecy of keys and assurance of purpose \cite{martineveryday}.  DTNs present fundamental differences with traditional networks that make goals of protocols sometimes unclear or not simple to define, key management is not the exception.

Considering routing in DTNs to illustrate the problem. The problem of defining the goals of routing in DTNs is analysed in many papers \cite{ivancic2009security,fall2003delay,araniti2015contact}. In traditional networks, the purpose of routing is to select the best path between a source a destination. Routing protocols utilise the network state to calculate the best path according to one or more metrics. In space DTNs, the lack of connectivity and long propagation delays make the goals of routing more complex. The topology could change faster than the time that requires disseminate the topology updates. Routing still consists of selecting the next best hop until the destination, but the concept of best path is constrained. In this case, maximise the probability of bundle delivery could be the goal. \cite{araniti2015contact}. 


Several key management schemes were evaluated for this report. Many of these schemes adopted different metrics and requirements. Thus, it seems that the problem is not well understood and the it is not requirements are not clear. The rest of the section present a detail study of the key management problem in DTNs.

\subsection{Analysis of current Key Management Solution}


The first question that arises is whether symmetric or asymmetric keys should be used to support a key management system.  The section  \ref{sec:space} concludes that symmetric keys are the most suitable solution for current space missions. 

For the Interplanetary Internet is expected a scenario similar to the terrestrial Internet \cite{rationale2010requirements}. Multi-hop communications and cross-organisational domains are enhancements of the current interplanetary communications. Also, policies, rights, hierarchy, and administrative groups form part of the overall solution. The network might scale to thousands of nodes, and different space agencies will participate in the network. Considering this scenario, it seems more suitable a key management solution similar to the Internet rather than one based on symmetric keys. Furthermore, symmetric keys might not be even acceptable in a cross-organisational domain \cite{ivancic2009security}.

The research effort so far is based on asymmetric keys and DTN working groups stated the need for automated systems for distribution of authenticated public keys and revocations mechanism \cite{templin-dtnskmps-00}. Although there are ciphersuites based on symmetric keys in the security specification BSP \cite{ietf-dtn-bpsec-07}, there is a common consensus that asymmetric keys will be the preferred method for the Interplanetary Internet. 

In public key cryptosystems, a node must have a copy of the destination public key to encrypt the message, and the destination nodes must have the source node public key to verify integrity. Key distribution is not a sensitive operation in the sense that information is not secret.  Common techniques  to obtain public keys are \textbf{pushing} directly from the key owner and \textbf{pulling} from a usually trusted directory \cite{martineveryday}.

Relatively low delays on the Internet allow pushing and pulling to be effective to obtain public keys. For instance, TLS uses the pushing approach to retrieve the destination public key. After obtaining the public key, a node needs to check the validity of certificates with a Certificate Authority CA. Popular options for this validation process are certificate revocation list CRL or the Online Certificate Status Protocol OCSP, but these options require interaction with the CA which might not be available in deep space communications. 

Here should be noted as well that TLS, one of the most popular cryptographic protocol on the Internet, uses a handshake between the parties to retrieve public keys and negotiate some parameters for the session. Negotiation based protocols are discouraged for DTNs \cite{fall2003delay,cerf2007delay}.

There are modifications of OCSP to improve the message overhead, but these are designed to alleviate the overhead on certificates receivers, but the overhead is transferred to the other side of the communication. In Secure/Multi-purpose Internet Mail Extensions (S/MIME) sender could encapsulate its certificate as meta-data, but the receiver is expected to validate the information. S/MIME remove the communication between nodes but still require on-demands interactions with a trusted authority. 

There are other alternatives to public key management based on certificates. Templin \cite{templin-dtnskmps-00} states that web of trust might be an option for some types of DTNs, but more research should be done. For space-based networks, a model without a trusted authority is not acceptable \cite{viswanathan-dtn-pkdn-00,burleigh-dtnwg-dtka-01,ivancic2009security}.  

Another alternative widely explored for public key management in DTNs is Identity Based Cryptography IBC. This approach uses a direct derivation of the public key from the node ID and eliminates the requirement of public certificates.  Even though IBC eliminates some problems inherent to the management of public certificates, it introduces other issues such as key escrow and key revocation. These problems can be addressed but no without adding considerably overhead which questioned the benefits of IBC for DTNs.


Farrell \textit{et al.} stated that IBC does not solve the problem of key management \cite{irtf-dtnrg-sec-overview-06}; however, Asokan \cite{asokan2007towards} argue that IBC indeed solves the problem and provides an improvement in confidentiality services but not for authentication. In any case, proposal based on IBC are considered in this work.   


\subsection{Taxonomy of key Management in DTN}

A key management solution for DTN in space has the same requirement as any key management system along with extra considerations for the hostile environment. This section focuses on understanding why the problem is different from terrestrial networks as the Internet, define the requirements, and classify the solutions. 



  Key management includes a wide range of processes which together provide security to cryptographic keys.  The problem of secure administration of cryptographic keys is complex but well-understood on terrestrial networks. Future space-based networks present two challenges to key management systems. Firstly, the network topology is dynamic; it consists of heterogeneous space nodes and planned communication links, opportunistic links might be available as well. Secondly, traditional cryptographic protocols are not suitable for this type of network; the latter problem is widely discussed in previous sections.% \ref{sec:dtn}.

Menesidou \textit{et al.} \cite{menesidou2017cryptographic} classify key management schemes into three categories depending on whether the solution deal or not with secure initialisation, key establishment, and revocation. Key establishment is divided again into two groups: two-party and group communication. The same approach is adopted in this report. As this project aims to choose the most suitable key management scheme for the Interplanetary Internet, assumptions and benefits are scrutinised. The analysis considers proposals which focus only on key establishment because this is the fundamental problem for DTN in space.

One objective of this project is analysing the proposed key management solutions for DTN. For that purpose, a set of requirements must be defined beforehand. Templin \cite{templin-dtnskmreq-00} presented for first time a set of formal requirements for a key management solution. Most of the requirements proposed by Templin are considered for this project. Some requirements such as multiple points of authority and no veto capabilities are excluded from the requirements defined in this project. Nevertheless, these omitted requirements are discussed in the section \ref{sec:evaluation}. Finally, two more requirements are added to the ones defined by Templin. 

It is worth to mention that these requirements complement the fundamental requirements for a key management system, and do not replace them. The new set of requirements REQ is listed below.
  


\begin{enumerate}
    \item Keys must be available when needed. The design must no relay on a query-response interaction between a source node and the trusted party or a destination node. This imply that credentials must be cached ``locally''. DTN nodes must store credentials locally or the access to them must be subject to low delay and rarely disrupted links using a sub-second one-way-light time.
    \item The system must be trustworthy. There must be a trust anchor in the system; nodes cannot accept information directly from other nodes, nor they can actively execute key management operations. Even if the operation is considered secure and nodes cannot fake messages, nodes could ``misbehave'' and drop updates or not collaborate with the protocol.
    \item A single point of failure is not acceptable. Nodes cannot depend only on the access to one entity,  some type of high availability must be present in the system.
    \item The system must support secure bootstrapping of nodes. The association between a node ID and its credentials must be certified before a node can use it in the DTN.
    \item The system must support delay tolerant key revocation. The system must have the capability to revoke credentials before the expiry date. 
    \item The systems must be scalable. A space DTN could grow to thousands of nodes and the system must be capable of handling this situation. 
\end{enumerate}


Adding to these requirement, there are characteristics that are highly desirable but not mandatory. These are 

low latency: The key management system should impose minimal latency in addition to the physical latency of the path
Low overhead: The key management system should not require more than a reasonable amount of messages exchange and storage. 
  

The fundamental problem for DTN in space is that space nodes require authenticated information without online interaction with a CA, distribution centre or other nodes. Information refers to keys or some form of identification. The latter is employed in solution based on non-interactive protocols or IBC. Thus, the availability of authenticated information is the main problem for DTN nodes, stated in REQ 1.




\subsection{Standardisation effort}

Even though DTN working groups have been active since 2007, a key management specification has been postponed for many years, mainly for its complexity \cite{rfc6257,irtf-dtnrg-sec-overview-06,templin-dtnskmps-00}. Farrell presents a high-level document for key management requirements \cite{farrell-dtnrg-km-00}, but this document does not establish concluding requirements for a solution. Only states that a solution must be compliant with the BSP, use well-known protocols and include solutions for manual keying, pre-shared keys and public keys.  


The first significant advance was a problem statement by Templin in 2014  \cite{templin-dtnskmps-00}. This internet-draft assumed that the solution involves some type of public key cryptography and an automated system for publication of certificates and revocation lists. More important, Templin affirmed that the system must be designed to continue operation in the presence of long delays and intermittent connectivity and traditional key management systems do not satisfy the requirements. The author suggests the use of one-time keys, so each bundle is encrypted with a symmetric key encrypted by the destination public key. 

Another Internet-Draft more detailed was published by the same author one year later \cite{templin-dtnskmreq-00}. The document proposed nine requirements for a key management solution. The requirements include no single point of failure, multiple points of authority and delay tolerant key revocation. The document also proposed four design criteria ensure that a solution matches with the requirements. For instance, the preference of a publisher/subscriber model instead of an online directory service and multiple sources of publications. 

Viswanathan \cite{viswanathan-dtn-pkdn-00} enumerates the possible communication patterns, data structure, architectural elements, and trust relationships.  This Internet-Draft provides a high-level solution for key distribution and revocations using the publisher/subscriber pattern, expanding the idea of previous documents.

Finally, a newly published document proposed the Delay-Tolerant Key Administration DTKA, which is a system for key management intended for the use in space-based DTNs. An overview of this system is presented in the next section.


\subsection{A Survey of Key Management in space DTN}

Aaditeshwar and Keshav \cite{seth2005practical} proposed one of the first key management systems for DTN. Their solution was designed for disconnected environments rather than space-based networks. They argue that disconnected nodes cannot efficiently use PKI because this requires queries to a central or replicated database, and certificates revocations list are unsuitable for the excessive delay. The authors proposed a solution based on HIBC for creating secure channels providing mutual authentication and key revocation. The Private Key Generator PKG is the trust anchor of the system, and there is one PKG per domain. Registration and establishment of system parameters require the exchange of several messages between the PKG, the distribution agent and DTN nodes, but this procedure is done only one time. The authors proposed USB key storage devices to distribute symmetric keys from PKG to final users. For revocation, they used time-based keys where the PKG updates periodically every node with a new time-stamped private key. 

This first key management system present some issues. Cross-domain communication requires a online look-up of the destination system parameters. Other option consist on the PKG of the source node generates the decryption key and then send to destination node PKG. The proposal does not meet the REQ 3, 4, and 5.


Another solution based on IBC is proposed by Aniket \textit{et al.}\cite{kate2007anonymity}. This solution is more efficient than the previous one because a source node does not need to look up the public parameters of DTN nodes in a different domain. Instead, it can use the root PKG parameters to communicate with any node in other domains. They also use a single-flow non-interactive key agreement scheme to exchange symmetric keys.

Although the last system is more efficient than the previous one, it fails to meet the same requirement, namely, key revocation, secure bootstrapping and single point of failure. However, the authors suggest a secret sharing scheme instead of using a single PKG to avoid a single point of failure. This paper is the only one that provides an alternative anonymity architecture, but this requires updating the domain PKG's master key, in consequence, it requires the renewal of all users private keys periodically. For space-based DTNs, anonymity is not stated as a requirement for space DTNs according to the CCSDS. \cite{book2011space}. In any case, the Bundle in Bundle Encapsulation \cite{ietf-dtn-bibect-00} already provides a way to achieve anonymity to some extent. 

Xie and Wang \cite{xie2013practical} propose a different approach based on bilinear pairing. They proposed a self-certified identity scheme and distributed key generation SCI-DKG that requires only an initial broadcast message rather than multiple interactions. Prior deployment, users obtain a digital ID card granted by a relevant authority and the system parameters. Each node generates its private key and distributes an initial message containing its credentials. When a node receives the initial message, it needs to verify if the signature matches with the credentials, then compute a secret sharing and extract the encryption parameters. Unlike the previous proposals, this scheme provides secure bootstrapping but revocation in not supported REQ 3. Another issue is that the system is not scalable REQ 7 and new DTN nodes receive information for other nodes and not from the trust anchor REQ 2.

Van Vesien presented a solution using bilinear maps over elliptic curves \cite{van2010dynamic}. This work is one of the few that explains in details how the key exchange protocol fit with the BSP specification. For instance, the author showed how the protocol could be used to exchange keys for HMAC-SHA 1 authentication, RSA digital signature, and AES encryption. Each node is members of one or more groups, and only nodes in the same group can non-interactively exchange keys. Initially, a group administrator calculates a value for each node using a master key and the node ID.  Then the administrator provides the calculated value to each node before deployment. Two hosts can non-interactively establish a shared secret using a symmetric mapping function, the value provided by the group administrator, and the parameters sent by the source containing key information. This proposal only focuses on key distribution, and there are no mentions of initialisation or revocation REQ 1 and 5. Scalability REQ 6 might be an issue because in practice a node can only be a member of two groups.


%For solving the key management problem, Jadhav \textbf{et al.} proposed using a time-evolving topology model and two channel cryptography to design a non-interactive key exchange protocol. This protocol assumes two channel, one insecure and one secure usually called Out-Of-Band OOB channel. The first stage is the bootstrap where the key owner present directly to the authority the authentication information and this could be easily checked. 

Lv \textbf{et al.} \cite{lv2014non} present another non-interactive key establishment solution for space DTNs. They used a time-evolving model for predicting contacts and exchange information such as key updates or revocation status. This model, combined with two-channel message authentication constitute the fundamental parts of the non-interactive key exchange protocol. In the bootstrapping stage, DTN nodes authenticate with a CA via an OOB channel providing secure initialisation. They claim that a proper OOB channel for a space DTN involves human interaction such as space missions authorities. In the exchange phase, a node sends its public key over an insecure channel like a radio signal to another node, then it calculates the hash of the public key and sends over the OOB channel. The receiver has to verify that the public key has not been modified. This solution requires an OOB channel in deep space. The authors suggest a laser emitter and a photo-sensor, but they stated that the design of an OOB channel is an open issue for space networks. Even if the development of laser technology is a promising technology for communication in space, it is unlikely that this will be used as a secure channel. Besides, there is no need to use the insecure channel if a secure OOB optical channel is available.  The key distribution and revocation is performed between DTN nodes in violation to REQ 2.  

Butha \textbf{et al.} \cite{bhutta2014efficient} present a key transport protocol to exchange symmetric keys which are compliant with the DTN architecture. They used a proxy signatures scheme which does not require a secure channel. This scheme is to provide the hop-by-hop authentication of the Bundle Authentication Block BAB. Each intermediate node validates the bundle, remove the BAB,  generates a new BAB using proxy signatures. The issue with this scheme is that the last version on the BSP eliminated the BAB.  The problem is that hop-by-hop authentication is difficult to achieve or impossible in some situations contemplated in the DTN architecture, so the overhead added by proxy signatures is not justifiable. REQ ??



The most recent proposal was published by the IETF \cite{burleigh-dtnwg-dtka-01} in 2018. The Architecture for Delay Tolerant Key Administration DTKA is a system of public keys management designed for space-based communication systems. This Internet draft is a full specification for the life-cycle of public keys in space-based DTN. The system adopt the publisher/subscriber model to proactively distributed public keys from the Key Agents to Key Users. The system also adopts a  multiple points of authority approach and no does not present single point of failure. 

Many Key Agents share the key authority role. DTN nodes register with the Key Authority through an OOB channel (for instance a physical channel). The Key Agents cooperate to generate bulletins with authenticated information such as new or revoked certificates. 

Key Agents distribute the bulletins using a t-n threshold scheme. Finally, nodes reassemble the bulletins and update their local database. All information in the bulletin is for future use, so nodes do not depend immediately on the bulletin. Scalability REQ 6 might be an issue for the DTKA. As this system is the one that satisfiy most of the requirements, a complete description is given at the end of the section.
%this topic is further discussed in the next section \ref{sec:evaluation}.  


Most of the proposal study in this section are based on traditional public certificates, Identity-based Cryptography or pairing cryptography. Rusch \textit{et al.}  \cite{rusch2017forward} proposed using puncturable encryption constructions for highly asynchronous scenarios. Their scheme ensures forward secrecy of messages at the cost of higher latency. In summary, recipients nodes repeatedly update the decryption keys to revoke decryption capabilities for selected time intervals, recipients or messages. The authors assumed that exist a PKI to exchange keys, but instead of RSA public keys, FSE public keys are used. No mention of initialisation or distribution is made on this work.


Group key management is an option for multicast communication. This project does not focus on group communication, but a brief analysis of some works is presented. Groups key management can be classified into centralised, decentralised and distributed. The most important security requirement for group key management are forward secrecy, backward secrecy, collusion freedom and key independence \cite{camtepe2005key}. 


Xu \textbf{et al.} \cite{xu2012chinese} proposed a group key management mechanism based on the Chinese Remainder Theorem for DTNs. The advantages are that the system does not need to broadcast messages when a node joins the domain, and only send one message when a node leaves the domain. A stateless scheme is used in this work to overcome delays and high error rates. For many to many communication, they add a lifetime to the group key to deal with users leaving the group. If a node does not receive the key update message at the corresponding interval, it can direct inquiry to a neighbour reducing the latency, but this is unacceptable for space-based networks REQ 2.

Another approach is proposed by Zhou \textbf{et al.} \cite{zhou2014autonomic} based on one-encryption multi-decryption key protocol for deep space. Authors claim that their scheme is more efficient than previous ones, but the join and leave operation could be done without the authority in violation of REQ 2. They argue that the operation has the property of forward-security and backward-security if the key independence condition holds.
 

Table \ref{table:summary} presents a summary of all key management solutions study in this work, it shows whether the proposals deal or not with the most challenging parts: initialisation, distribution, and revocation. It shows as well if each solution is for two-party or group distribution and the method or scheme used.  

Table \ref{table:summaryREQ} shows which requirements satisfies each solution for two-party communication. Note that an ``x'' mark appears in the case the system does not meet the requirement or does not address the problem at all.    

%The table illustrated the proposals and the part of the problem that these schemes address, specifically, registration, establishment and update/revocation. 

\begin{landscape}

\begin{table}[hbt]
\begin{tabular}{lccccc}
 & \multicolumn{5}{l}{\cellcolor[HTML]{000000}{\color[HTML]{EFEFEF} Classification of key management systems}} \\
Ref. & \multicolumn{1}{l}{\cellcolor[HTML]{C0C0C0}Init/Reg} & \multicolumn{1}{l}{\cellcolor[HTML]{C0C0C0}Dist.} & \multicolumn{1}{l}{\cellcolor[HTML]{C0C0C0}Rev.} & \multicolumn{1}{l}{\cellcolor[HTML]{C0C0C0}Individual/Group} & \multicolumn{1}{l}{\cellcolor[HTML]{C0C0C0}\centering Method} \\
\cellcolor[HTML]{C0C0C0}Seth and Keshav\cite{seth2005practical}  & $\surd$ & $\surd$ &  & indiv & HIBC, HIBE and HIBS \\
\cellcolor[HTML]{C0C0C0}Aniket \textit{et al.}\cite{kate2007anonymity} & $\surd$ & $\surd$ &  & indiv & HIBC \\
\cellcolor[HTML]{C0C0C0}Xie and Wang\cite{xie2013practical} & $\surd$ & $\surd$ &  & indiv & distributed secret generation, bilinear pairing, self-certify identity \\
\cellcolor[HTML]{C0C0C0}Van Vesien \cite{van2010dynamic} & & $\surd$ &  & indiv & IBC bilinear maps over elliptic curves \\
\cellcolor[HTML]{C0C0C0}Lv \textit{et al.} \cite{lv2014non} & $\surd$ & $\surd$ &  & indiv &  time evolving model, two-channel cryptography \\
\cellcolor[HTML]{C0C0C0}Butha \textit{et al.} \cite{bhutta2014efficient} &  & $\surd$ &  & indiv & certificates, proxy signatures \\
\cellcolor[HTML]{C0C0C0}Andrade and Pessoa \cite{de2016fully}  & $\surd$ & $\surd$ &  & indiv & decentralised, PGP, signatures chains \\
\cellcolor[HTML]{C0C0C0}Burleigh \textit{et al.}\cite{burleigh-dtnwg-dtka-01} & $\surd$ & $\surd$ & $\surd$ & indiv & certificates, bulletins \\
\cellcolor[HTML]{C0C0C0}R\"{u}sch \textit{et al.}\cite{rusch2017forward} &  &  & & indiv & puncturable encryption, forward secure keys \\
\cellcolor[HTML]{C0C0C0}Butha \textit{et al.} \cite{bhutta2016public} &  &  & $\surd$ & indiv & certificates, CRL, hash tables \\
\cellcolor[HTML]{C0C0C0}Xu \textit{et al.} \cite{xu2012chinese} & $\surd$  & $\surd$ & $\surd$ & group & chinese theorem remainder \\
\cellcolor[HTML]{C0C0C0}Zhou \textit{et al.} \cite{zhou2014autonomic} & $\surd$ & $\surd$ & $\surd$ & group & logical key hierarchy , one-encryption multiple-decryption keys \\ \end{tabular}
\caption{Summary of key management systems}
\label{table:summary}
\end{table}

\end{landscape}


\begin{table}[hbt]
\begin{tabular}{@{}lccccccc@{}}
 & \multicolumn{7}{l}{\cellcolor[HTML]{000000}{\color[HTML]{EFEFEF} key Management Requirement}} \\
 Ref. & \multicolumn{1}{l}{\cellcolor[HTML]{C0C0C0}REQ 1} & \multicolumn{1}{l}{\cellcolor[HTML]{C0C0C0}REQ 2} & \multicolumn{1}{l}{\cellcolor[HTML]{C0C0C0}REQ 3} & \multicolumn{1}{l}{\cellcolor[HTML]{C0C0C0}REQ 4} & \multicolumn{1}{l}{\cellcolor[HTML]{C0C0C0}REQ 5} & \multicolumn{1}{l}{\cellcolor[HTML]{C0C0C0}REQ 6} & \multicolumn{1}{l}{\cellcolor[HTML]{C0C0C0}REQ 7} \\
\cellcolor[HTML]{C0C0C0}Seth and Keshav \cite{seth2005practical}  &  & $\surd$ &  & $\surd$ &  &  & $\surd$ \\
\cellcolor[HTML]{C0C0C0}AniKet \textit{et al.} \cite{kate2007anonymity} & $\surd$ & $\surd$ &   &   &  &  & $\surd$ \\
\cellcolor[HTML]{C0C0C0}Xie and Wang \cite{xie2013practical} & $\surd$ & & $\surd$  & $\surd$  & &  & $\surd$ \\
\cellcolor[HTML]{C0C0C0}Van Vesien \cite{van2010dynamic}  & & $\surd$ & & $\surd$  & & & $\surd$ \\
\cellcolor[HTML]{C0C0C0}Lv \textit{et al.} \cite{lv2014non} & $\surd$ &  &  & $\surd$  &  &  & $\surd$ \\
\cellcolor[HTML]{C0C0C0}Butha \textit{et al.} \cite{bhutta2014efficient} & $\surd$ & & &  & & $\surd$ & $\surd$ \\
\cellcolor[HTML]{C0C0C0}Andrade and Pessoa \cite{de2016fully} & $\surd$ &  & $\surd$  &  & & $\surd$ & $\surd$ \\
\cellcolor[HTML]{C0C0C0}Burleigh \textit{et al.} \cite{burleigh-dtnwg-dtka-01}   & $\surd$   & $\surd$   & $\surd$ & $\surd$ & $\surd$  &  & $\surd$  \\
\cellcolor[HTML]{C0C0C0}R{u}sch \textit{et al.}\cite{rusch2017forward}   & $\surd$  & $\surd$ &  & $\surd$ &  &  &   \\
\end{tabular}
\caption{Key Management Systems VS Requirements}
\label{table:summaryREQ}
\end{table}





 \newpage


\section{Discussion and Challenges}
\label{sec:evaluation}

One critical consideration before choosing the best key management system is understanding how the Interplanetary Internet will operate. Namely, the infrastructure available, the routing model, the trust model and certificate authority. Comments from professionals in the space industry were constructive to review these topics.


\subsection{An operational Interplanetary Internet}

The first point to discuss is the infrastructure available to route the bundles. The question is whether an ``interplanetary backbone'', formed by relay satellites,  or a peer-to-peer approach is most likely to be deployed. In the former scenario, DTN nodes have to wait for an opportunity contact with one DTN backbone node to send messages. In the peer-to-peer style, nodes use other nodes to route bundles, in an ad-hoc style. In any case, the contact planning determine routing policies and access rights, this is documented in the BSP and \cite{cerf2007delay}. 


The belief in the space industry is that both models will have their place in the Interplanetary Internet. The interplanetary backbone is bound to emerge at some point in the future, but it will always be situations where DTN nodes will send bundle across the network using other peers. 

Some contact planning authority is required to generate and distribute routing updates. The information has to be authenticated, just like keys. Some researchers suggest that secure routing in space DTN overlaps in some extent with key management \cite{}. For some part of the network, the information will come from a central authority, but at the edges of the interplanetary network, the contact planning will be ad-hoc or local. As a result, the contact planning problem is more complicated than key distribution, and the problems are disjoint.  


Space exploration has shown that cooperation among nation-states is possible. A natural question is whether space agencies can continue this cooperation to simplify the key management or not. The question comes in the context of choosing the key management model. This problem is arguably one the most complex on terrestrial networks and the same issue must be solved for space-based networks. A new complication to this scenario is that it is expected that in the next 10 years, the private sector will play a major role in space exploration. 


Initially, the problem resides in agree on the number of CA. In the first option, space agencies agree to trust a single CA, this is by far the best scenario. The multiple authority approach is the second option. For instance, each space agency could administrate one ``CA node'' and all together perform the CA role. 

Indisputably the first option is hard to achieve but space exploration has shown that collaboration between nations is possible and this could motivate the single CA approach. A reasonable trust relationship is required is space agencies are going to share infrastructure in space. Ivancic \cite{ivancic2009security} mention that the Interplanetary Internet could look like a closed network open only to trusted parties. The second option requires some kind of cross-organisational certification which includes well-known problems, for example how and why should a node trust in a CA from another application domain. 


Surprisingly, according the expert in the space industry, these questions, infrastructure, routing model and key management models have the same answer: All option will be realised in the Interplanetary Internet. The network topology will be a bunch of disjoint networks rather than a single Solar System Internet to the disadvantages of all participants. Some space agencies will cooperate and form a big infrastructure, others will be close, just as their corresponding nation. Even one space agency could participate in a joint network with other agencies but remain some assets to be used only for them.  




"The Interplanetary Internet named was deliberately coined to suggest a far-future integration of space and terrestrial communication infrastructure to support the migration of human intelligence through the Solar System" \cite{burleigh2003interplanetary}. Deploy a key management system with substantial differences of the Internet (TLS certificates) which does not present sustancial benefits is counterproductive. 


Symmetric keys may be a requirement even for future space missions. There are sensitive operations which require symmetric keys between the spacecraft and the mission control centre. Examples include recovery operations, low-level telemetry and low-level telecommands.  However, there are some ideas on the DTN working groups involving the use of applications on the top of the bundle layer to handle these operations. 

There are public key management system without a trusted party like group based KM, PGP or block-chain based model. The problem is that revocation of public keys becomes impossible or unsuitable for a real scenario. The need of a trusted party acting as Certificate Authority becomes a requirement. Another problem is the message overhead in these model, something that it is unlikely to be acceptable in any king of delay tolerant network. 



the general concern of DTN is ``How to address the architectural and protocol design principles
arising from the need to provide interoperable communications with and among extreme and performance-challenged environments, where continuous end-to-end connectivity cannot be assumed'' \cite{caini2011delay}.

\subsection{IETF Requirements}

Templin \cite{templin-dtnskmreq-00} sketches 9 requirements (REQ) and 4 design criteria for a delay tolerant key management system. The design criteria list is defined to enforce the key management system requirements. The requirements not considered in this project are multiple points of authority (REQ 4) and no veto capabilities (REQ 5). These requirements are discussed below. 

Multiple point of authority and no veto capabilities are two side from the same coin. The idea is that DTN nodes must not be forced to trust in a single KA and no single KA compromised can spoil the protocol. Considering the implementation aspects of the Interplanetary Internet discussed before, these requirements fit only for a subset of interplanetary networks, in consecuence, it can not be stated as a fundamental requirement but it can be considered as desirable property in particular situations. 

The DTKA uses multiple Key Agents, a consensus protocol to agree the bulletins content, and distrubutes the risponsability of parts of the bullentin to different KAs to avoid a single point of failure. Other solutions, for instance based on ID-PKC, could use a secret sharing scheme for the same purpose. Both provide robustness to the system at the cost of computational and communication overhead which are not suitable to every space DTN. 


The REQ 3 states that the system must not introduce a single point of failure. This requirement is reasonable for two reason. First, high availability of the key management system, in the case of a failure in the system. Second, the planets motion make the space DTN a disconnected network, some part of the network will always be unreachable from the trusted party. The document mention that ``DTN nodes cannot always depend on receiving information from any single key authority node'' but the same concept apply in the other way, when DTN nodes have to sent information to the authority. 

The REQ 6 states that the system must bind a public key with a node Id, this is a fundamental requirement for key management system, so it is assumed that all key management systems must provide this requirement. 

The publisher-subscriber pattern seems to be the best option to distribute keys assuming key management based on public certificates. The multiple points of authority will be deploy in some application domains but a single point of authority will be deployed in others. In the latter case, a much simpler initialisation and distribution mechanism will be used. 


Most systems studied in the previous section does not address this problem. A common idea for ID-PKC is a secret sharing scheme, Aniket {et al.} \cite{kate2007anonymity} proposed exactly this solution. This solve in some extend the problem but introduces additional requirements for infrastructure and messages exchange \cite{al2003certificateless}. The DTKA provides the option of multiple Key Agents in one application domain but in its current specification, the solution contemplate the problem when the Key Agents have to send information but not when they have to receive. All Key Agents must receive the request messages from KO or KU. This requirement could be relax to deal with situation such as a Key Agent down.    


Identity based cryptography: is impractical to generate new keys for short time periods in DTNs.
Paper de Asokan, Nokia. IBC-based system try to solve the problem of key distribution and revocation list, eliminating the needs of having them. However, it is not clear the practical success as this solutions introduce many other problems. Asokan \cite{asokan2007towards} present a design to authenticate and encrypt messages using IBC and traditional PKI. The conclusion of his work is that IBC provide less message overhead for encryption but not for authentication, all these without considering the problems that ID-PKC introduce. 

\subsection{Revocation List}
 \newpage

\section{Conclusions}
\label{sec:con}

For interplanetary communication, the presence of celestial bodies affects the availability of links and considerably distances result in long propagation delays. For these reasons, traditional protocols are not suitable for this kind of environments, and significant work has been done to define a delay tolerant architecture which overcomes these problems.   


The first days of space exploration required relatively simple communication systems; the communications were point-to-point between spacecraft and the operation control centre, and mission operated individually. The quick development of telecommunication made difficult the adoption of standard technologies. As a result, customs equipment, interfaces, and protocol were the norm. 


Future space missions envision a shift from point-to-point communication towards a multi-hop communication, and from an isolated environment to a cooperative model. The CCSDS is the organisation which lead the effort to standardise protocols for space-based systems. The Bundle Protocol is recognised as the strongest candidate to support in-space internetworking, as it provides reliability in the presence of network partition and long delays. 

Security is an essential feature of the Bundle Protocol for space missions, integrity and confidentiality services prevent unauthorised access to scarce resources, especially in the space segment. However, security specifications assume that bundle nodes are in possession of cryptographic keys.     

Key management is recognised as one of the most challenging problems in Delay Tolerant Networks, and there is no common consensus on which solution suit best to future interplanetary networks. Bulk encryption and symmetric keys are commonly used to secure communication in space, but his model does not meet the requirement of future interplanetary networks, and an "Internet-style" model is expected for key management. This work presented an analysis of key management systems for the envisioned Interplanetary Internet. 

%This work classifies key management systems depending on whether the scheme deals with secure initialisation, key establishment and revocation, which are the most challenging parts of the life-cycle.  

It is well-known that the problems of public key management lie in the deployment and administration of the infrastructure to support cryptographic keys rather than cryptographic protocols. The first challenge was to understand why this problem is different from terrestrial networks and define the requirements for a suitable solution. 


First, this work identifies the access to authenticated information without online interaction to any party as the fundamental requirement for key management. Network partition is a typical situation in space DTNs, and the connection to a distribution centre or the other parties cannot be assumed. Moreover, the communication paths between  (sender,receiver), (sender, trusted party), and (receiver, trusted party) are not uniform, resulting in variable delays. 


Second, this work defines seven requirements to evaluate key management proposals. These requirements encompass the availability of keys, trust anchor, availability of the system, secure initialisation, revocation, scalability and compatibility with Bundle Protocol. None of the evaluated solutions satisfies all requirements, but the Delay-Tolerant Key Administration system  (an IETF Internet-Draft) achieves six of them, leaving scalability as an issue to be improved. 

ACA EXPLICAR BREVEMENTE LAS MEJORAS PARA EL DTKA. 
  


 
%Lower operational cost, highest data return and more floxible system that can support the growthing number of spacecraft are other motivations. 

 

%``The general concern of DTN is How to address the architectural and protocol design principles arising from the need to provide interoperable communications with and among extreme and performance-challenged environments, where continuous end-to-end connectivity cannot be assumed'' \cite{caini2011delay}.

%A space DTN is a particular case of delay tolerant network. By far, is the most extreme scenario where delays could be extremely long and the communications get disrupted by the planets orbit and motion.





% This work has been about building 

% Significant work has been done
% It is important to consider 
% The most significant limitation to the 
% As has already been made clear
% It is also important to investigate 

 \newpage


\bibliographystyle{plain}
\bibliography{bibliography}
\end{document}


