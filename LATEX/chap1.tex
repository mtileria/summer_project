\section{Introduction}


-Different research groups and organisation CCSDS, IPN
-experiments
-why no tcp and others
-other protocols: LTP, CFDP, 
-security


Cerf \textit{et at} \cite{cerf2007delay} define the Interplanetary Internet as ``a communication system design to provide Internet-like services across Interplanetary distances in support of deep space exploration''. The original motivation of the Delay Tolerant Architecture was the design of the Interplanetary Internet Architecture. Then, the specification evolves to cover a wide range of delay tolerant environments like wireless sensor networks, satellites networks, military tactical networks and others, and end with the specification of the Bundle protocol \cite{scott2007bundle} and a series of related documents. 


Cryptographic Key Management is recognised as the most difficult problem in DTNs. For many years, research and working groups from IETF and other organisation left the problem unsolved and assumed that there was a reasonable key management solution that can be used to generate, update and distribute keys among DTN nodes. 

Key management is a central part of any secure communication system and plays a critical role in overall security. Furthermore, key management is recognised as a difficult problem in a connected network, and a disrupted network subject to long delays make the problem even more complicated.  


The future space exploration envisions multiple space nodes that communicate with each other and not only between space and Operation Centre (OC). A space node could be any human-made object send into deep space in support of space mission, for instance, satellites, orbiters, landers, rovers or any small robot. An immediate consequence of the new communication model will be an increase in the number of interconnected systems, data transfer and connectivity. The current architecture is not designed to support such expansion, and an upgrade will be necessary. A similar situation occurred in the early days of telephone systems and switchboards. As the number of users grew, it was not possible human circuit switching anymore, and the system had to evolve into an automated switching systems \cite{rationale2010requirements}. The space communication architecture requires a transformation from point-to-point communications toward a flexible architecture capable of managing the new communication requirements of future space missions. 

