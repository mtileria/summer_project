\section{Introduction}

Cerf \textit{et al.} \cite{cerf2007delay} define the Interplanetary Internet as ``a communication system envisioned to provide Internet-like services across Interplanetary distances in support of deep space exploration''. The original motivation of the Delay Tolerant Architecture was the design of the Interplanetary Internet Architecture. However, the work evolves to cover a wide range of delay tolerant environments like wireless sensor networks, satellites networks, and military tactical networks. 
%and end with the specification of the Bundle protocol \cite{scott2007bundle} and a series of related documents. 

Space-based communications present new challenges to protocols designers.  The vast distance between a mission operation centre and a spacecraft in deep space makes the propagation delay considerably long. The propagation delay range within the Solar System is 1.5 seconds to  2.5 hours. The second major problem is disrupted communications due to the lose of line-of-sight of space objects because of planets motion. These particular properties of space-based communications make well-known protocols unsuitable for deep space communication, including cryptographic protocols and key management. 


The Delay Tolerant Architecture \cite{cerf2007delay} and the Bundle protocol \cite{scott2007bundle} were defined by to overcome these problems. The Bundle protocol operates on the top of lower layer protocols creating a store and forward overlay network \cite{ivancic2009security}. The current work is being done by academic researchers, and DTN working groups on the Internet Research Task Force IRTF, and the Consultative Committee of Space Data Systems CCSDS. The latter organisation has the mission of standardise spaced-based communication protocols since 1981.

%s a multi-national forum for the development of communications and data systems standards for spaceflight.
%The Bundle protocol is a end-to-end message-oriented overlay 

The CCSDS recognises that all types of space missions are potential target of malicious attackers.  Security specifications \cite{ietf-dtn-bibect-00,ietf-dtn-bpsec-07,rfc6257} define integrity and confidentiality services for Bundle protocol. Security blocks and mandatory ciphersuites are defined in these specifications as well.  Cipher-suites based on symmetric and asymmetric are considered in the documents. However, no key management is specified.   

Key management is a central part of any secure communication system. Generate, distribute and update keys is a challenging problem in connected networks, but in DTNs the problem is even more complicated. Furthermore, key Management is recognised as one of the most difficult security problems for DTNs  \cite{menesidou2017cryptographic}.

Nowadays, bulk encryption is used to secure communication between spacecraft and its mission control centre. The use of symmetric keys is a efficient solution considering the limited resources and the predominant point-to-point communication in current missions. However, the Interplanetary Internet envisions a network of multiple space nodes that send messages using multi-hop communication. Moreover interoperability between space agencies makes more complicate the deployment of symmetric keys and new key management solution are needed.


For many years, DTN working groups omitted the key management problem, nevertheless, many solutions for all types of DTNs were proposed, including space-based DTNs. Finally, The IETF published a draft for a Delay Tolerant Key Administration in 2018 \cite{burleigh-dtnwg-dtka-01}. Section \ref{sec:survey} presents a study of these proposals.   


This project studies the key management problem considering the hostile deep space environment and the future communication model. 

The section \ref{sec:space} provides an overview of the communication in space missions. Section \ref{sec:dtn} explain the bundle protocol, the security specifications and key management requirements. A survey of the proposed solution is presented in section \ref{sec:survey}. Section \ref{sec:evaluation} present an evaluation of the most promising solution and suggest a way to improve them. Final considerations and conclusions are given in section \ref{sec:discussion}. 


%The current architecture is not designed to support such expansion, and an upgrade will be necessary. A similar situation occurred in the early days of telephone systems and switchboards. As the number of users grew, it was not possible human circuit switching anymore, and the system had to evolve into an automated switching systems \cite{rationale2010requirements}.

%Space communication architecture requires a transformation from point-to-point communications toward a flexible architecture capable of managing the new communication requirements of future space missions \cite{rationale2010requirements}. 


