\section{Introduction}

Cerf \textit{et al.} \cite{cerf2007delay} define the Interplanetary Internet as ``a communication system envisioned to provide Internet-like services across Interplanetary distances in support of deep space exploration''. The designing of the Interplanetary Internet was the original motivation for a delay-tolerant architecture. However, the work evolves to cover a wide range of Delay Tolerant Networks (DTNs) like wireless sensor networks, satellite networks, and military tactical networks. 
%and end with the specification of the Bundle protocol \cite{scott2007bundle} and a series of related documents. 

Space-based communications present new challenges to protocols designers. The first problem is the vast distance between mission operation centres and human-made objects in deep space, what makes the propagation delay considerably long. For instance, the propagation delay in the Solar System is in the range of 1.5 seconds to  2.5 hours. The second major problem is disrupted communication channels due to planets motion. These particular properties of space-based communications make well-known protocols unsuitable for deep space communication, including cryptographic protocols and key management.  Protocols based on negotiation between the parties, like the Transport Layer Security (TLS) handshake, are discouraged for interplanetary communications \cite{fall2003delay,cerf2007delay}.


The Delay Tolerant Architecture \cite{cerf2007delay} and the Bundle protocol (BP) \cite{scott2007bundle} were defined by to overcome these problems. The Bundle protocol operates on the top of lower layer protocols creating a store and forward overlay network \cite{ivancic2009security}. Note that spacecraft, satellites, space station, landers, or rovers could contain one or more bundle nodes.


The current work is being done by academic researchers, DTN working groups on the Internet Research Task Force (IRTF), the space industry, and the Consultative Committee of Space Data Systems (CCSDS). The latter organisation has the mission of standardise spaced-based communication protocols since 1981.

%s a multi-national forum for the development of communications and data systems standards for spaceflight.
%The Bundle protocol is a end-to-end message-oriented overlay 

The CCSDS recognises that all types of space missions are potential target of malicious attackers.  Security specifications \cite{ietf-dtn-bibect-00,ietf-dtn-bpsec-07,rfc6257} define integrity and confidentiality services for Bundle Protocol, as well as security blocks and mandatory ciphersuites for cryptographic protocols. The ciphersuites include options for symmetric-key algorithms and Public Key Cryptography (PKC); however, no key management is defined. 

Key management is a central part of any secure communication system. Generate, distribute and update keys is a challenging problem in connected networks, but in DTNs the problem is even more complicated. Furthermore, key Management is recognised as one of the most difficult security problems for DTNs  \cite{menesidou2017cryptographic}.

Nowadays, bulk encryption is used to secure communication between a spacecraft and its mission control centre. The use of symmetric keys is an efficient solution considering the limited resources and the predominant point-to-point communication in current missions. However, the Interplanetary Internet envisions a network of multiple space nodes that communicate with each other using multi-hop routes. Moreover, interoperability between missions from different space agencies make the current model unsuitable and new solutions are required for space missions.


For many years, DTN working groups assumed there was a suitable key management solution and focused on other security aspects of the Bundle Protocol. The Bundle Security Protocol (BSP) \cite{ietf-dtn-bpsec-07} is built on the assumption that DTN nodes already have access to authenticated copies of public keys or share a symmetric key. Nevertheless, researchers proposed several key management schemes for space-based and general DTNs, but there is no consensus on the best solution. This project studies the key management problem considering the hostile deep space environment and the future communication model for space missions. 

%Finally, The IETF published a draft for a Delay Tolerant Key Administration in 2018 \cite{burleigh-dtnwg-dtka-01}. Section \ref{sec:survey} presents a study of these proposals.   

Firstly, section \ref{sec:space} provides an overview of the current communication architecture for space missions. Section \ref{sec:dtn} explains the Bundle Protocol, the security specifications and key management. A survey of the proposed solutions is presented in section \ref{sec:survey}. Section \ref{sec:evaluation} presents an evaluation of the most promising solutions and suggest a way to improve them, as well as challenges and discussions on selected topics. Finally, conclusions are given in section \ref{sec:con}. 

%The current architecture is not designed to support such expansion, and an upgrade will be necessary. A similar situation occurred in the early days of telephone systems and switchboards. As the number of users grew, it was not possible human circuit switching anymore, and the system had to evolve into an automated switching systems \cite{rationale2010requirements}.

%Space communication architecture requires a transformation from point-to-point communications toward a flexible architecture capable of managing the new communication requirements of future space missions \cite{rationale2010requirements}. 


