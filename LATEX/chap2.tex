\section{Deep Space Communications}

There are fundamental differences between space-based communication and traditional communication on the earth. Understand the physical limitations of communication in deep space is fundamental at the time of design secure protocols. This section gives a review of the principal components of space-based communication, the evolution of the communication model for space missions and provide a security overview of space-based networks.

The launch of the Sputnik 1 satellite gave origin to the Space Race. Human spaceflights and unmanned robotic space probes were developed to support space exploration missions. A communication system was needed to send commands from a mission control centre and receive information from spacecraft. The original idea was simple: the mission control centre sends a Radio Frequency (RF) signal to a spacecraft, known as telecommand, the spacecraft receives the signal via its antenna and process the telecommand, then the spacecraft replies with telemetry command or science data if necessary. As space missions grew in complexity and distance to the earth, communication systems had to evolve to keep the pace. 

A simplified diagram of a space mission architecture is shown in the figure. 

INSERT FIGURE!!!!
INSERT FIGURE!!!!
INSERT FIGURE!!!!

In this example, the satellite and the lunar rover form part of the space segment. The space-link segment is the connection between the ground station and the satellite.  The ground station and the mission control centre form the ground segment, sometimes the mission control centre is called Operational Control Centre OCC, and it has the control of the objects in deep space.

 The most significant differences of space-based communication are the long propagation delay due to the speed of light and the lack of end-to-end connectivity caused by planets motion, resulting in the disruption of connectivity between a mission control centre and space objects. These properties make well-known protocols for terrestrial networks unsuitable for space communications \cite{fall2003delay}.


The distance of an object to the earth determines the type of the communication between spacecraft and ground stations. Near-earth communications present low delay and intermittent connectivity. Satellites in geosynchronous orbit are subject to low delay but continuous connectivity. Earth-Moon communication is a particular case, the round-trip delay is approximately 2.5 seconds, and there is a disruption in the communication due to the motion of the earth and the moon. Deep space missions beyond the moon present the worst case: extremely long propagation delay and disrupted communication. For instance,  a signal from the farthest human-made object in deep space (Voyager 1) takes over 19 hours to reach the Earth. Signals sent by Mars rovers take between 7 and 20 minutes to reach Earth, depending on Earth-Mars relative position. Moreover, a rover has to be in line-of-sight of an orbiter or the Earth to send or receive data, producing disrupted communication. 



There are significant limitations to the current communication model in space missions. Firstly,  the communication between a spacecraft and its mission control centre is mainly point-to-point. Second, each mission operates independently, thus the cooperation is almost nonexistent, although Mars exploration is an exception. An example is bespoke communications protocols; a space mission typically focuses its resources on immediate problems and subsequent mission have to ``reinvent the weel'' \cite{burleigh2003interplanetary}. 

However, over the last years, there is an effort from CCSDS and other bodies like IETF to standardised communication protocols for space missions. Examples include Space Packet Protocol and the Bundle protocol for space communication, both developed by CCSDS.% \cite{standard2010ccsds}. 

Mars exploration motivated the use of orbiters as hop relay nodes between rovers and mission control centre back on Earth. This relay mechanism was the first advance towards a packet-switched network; however, there is no proper inter-networking yet under this configuration. Mars rovers use available orbiters as intermediate hops, but there is no addressing scheme, no classes of data and no proper network layer. These limitations will restrict operations of future missions which will require more communication between space and ground segment \cite{rationale2010requirements}. 


The experience has shown the advantages of multi-hop communication over point-to-point communication. Some of them are: increase in science data return, lower power and hardware requirements for nodes, and more contact opportunities \cite{rationale2010requirements}. Besides, it is predicted that future space mission will operate in an environment of interoperability and cross-support across space agencies. Spacecraft, satellites, rovers and other human-made objects will act as a network of space-based entities as shown in figure XXX.




INSERT FIGURE!!!


\subsection{Security in Space Communications}



Nowadays,  cyber threats apply to all kind of information systems especially the ones administrated by nation-states. Space systems are becoming more interconnected to terrestrial systems.   In this environment, space missions could be the target of malicious attackers.  In the past, only military missions were highly protected, but this is not valid anymore, and all mission requires a level of security \cite{book2006security}.

As the traditional communication model in space missions is point-to-point, data links have been secured using bulk encryption. Although this technique is simple, requires special techniques and hardware to be deployed on both end of the space-ground segment.   Bulk encryption is simple but not scalable, multi-hop communication and interoperability among space agencies exclude bulk encryption for the Interplanetary Internet. Ivancic \cite{ivancic2009security} mention potential problem in the US for international interoperability because The International Traffic in Arms Regulations might have jurisdiction.

The CCSDS present a list of threat sources and types of threats applicable to space missions \cite{book2006security}. The adversaries include terrorist, criminals, foreign intelligence services, computer hackers, and commercial competitors. There are other sources as insiders and structural which require a different attention. Threats could be active or passive. Examples of the first are jamming, unauthorised access, masquerading, Denial of Services, and examples of passive threats are tapping, traffic analysis. The outcome is that threats for space missions should be considered the same way as any other information system.


For many countries, satellites already form part of critical infrastructures like navigation, weather study, and disaster response.  As state before, future missions will require interoperability between space nodes that might be administrated by different agencies. Therefore, space mission planners must consider the implementation of secure communications protocols whether the nodes are in orbit around Earth or exploring remote places in the solar system  \cite{book2006security}.


It is worth to mention that threats could apply to any segment of space missions: space segment, ground segment, space-link segment. Two important points should be noted in the diagram XXX which might affect security.  Firstly, the mission control centre and the ground station are not in the same physical location but requires a secure connection between them. Secondly, the network infrastructure on the earth could belong to one entity and the mission control centre to another.  

An example of the previous idea is the Mars Exploration Mission, which has its mission control in NASA Jet Propulsion Labs but the rovers and orbiters use the Deep Space Network ground stations. This situation suggests that security for the Interplanetary Internet should consider interoperability not only between nodes from different agencies, but it should consider interoperability between the Internet and space segment protocols. 

 %In \cite{book2012architecture}, CCSDS present security requirements for five space mission profile. These profiles are human spaceflight, earth observation, communication, scientific, and navigation. For instance, humans spaceflights present all security requirements but also ``safety-of-life'' and privacy issues. The security of earth observation, navigation, and communications missions vary depending on the information value and the relative position to the earth. Some missions require security for telemetry, telecommands, and payload communication, others only need to secure a subset of those. 
 
 Five space mission profile are presented by CCSDS \cite{book2012architecture}. Each profile demands different security requirements. The profiles are human spaceflight, earth observation, communication, scientific, and navigation. Some missions require security for telemetry, telecommands, and payload communication, others only need to secure a subset of those. Science missions could have spacecraft or robots deployed in remote locations in the solar system. Extra factor should be considered for these missions that could influence security. For instance fault tolerance, the use of intermediate relay nodes and significant mission lifetime.  In multi-organisational missions, payloads and data may belong to different space agencies. Relay satellites and endpoints may be administrated by different agencies as well.
 
 %For instance, humans spaceflights present all security requirements but also ``safety-of-life'' and privacy issues. The security of earth observation, navigation, and communications missions vary depending on the information value and the relative position to the earth. 
 
 %Science missions are more complicated than the previous,  the distance to the earth change requirements dramatically. For interplanetary missions, there are extra factors that should be considered at the time of implementing security, for instance, communication delay, discontinuous communication, fault tolerance, ability to use intermediate nodes (planned and unplanned), significant mission lifetime. The last point to consider is 
 
 %A clear example is optical communications in space, state of the art techniques are not competitive in mass and power performance against radio frequency RF communication, and there are several projects ongoing to meet the performance goals required by mission planners. It remains to see how different profiles could fit in a single key management scheme.
 



\subsection{Key Management for current Missions}

In current missions, key management complexity is low. There are mainly two entities involve: the mission control centre or operation control centre OCC and the spacecraft. The OCC is responsible for generation, managing and revoking keys. In the context of human-crewed missions, the human intervention in spacecraft is limited. Optionally, the ground station could take part in the key management in the case it is working as security gateway and user facilities may be present if payload data have to be disseminated \cite{book2011space}. 

Differences between current mission and future missions:

- key generation procedure affects only the ground segments. 

It is recommended to store keys in secure storage such as smart card and physical protected place. Master key is used to exchange traffic protection keys TPKs. this process may takes place inside the operational network. 

Initial pre-distribution of symmetric keys. Master keys are used to generate or exchange new keys of a lower hierarchy as a traffic protection key. In the case that a master key is used as encryption key, all key management operation are conducted before launch. Could be more than one master key \cite{book2011space}.

Many environmental properties influence a key management solution in space missions \cite{book2011space}. Asymmetric channels and bandwidth restriction, propagation delay, intermittent connectivity, remote location, limited computational resources and memory. These limitations imply that symmetric key cryptosystems are more suitable in space missions. 

For the Interplanetary Internet is expected a full constellation scenario in which each spacecraft, robot,  ground station, OCC, and end user facilities are individuals nodes in a global network using the Bundle Protocol. Under this conditions, it seems more likely a scheme similar to terrestrial Internet as not only keys are in consideration, policies, rights, hierarchy,  and administrative groups should be consider. Moreover, the network might scale to thousands of nodes, and symmetric keys deployment becomes more difficult considering that different organisations participate in the network.   
Heterogeneous nodes will operate in the Interplanetary Internet, and some of them may have an active role in the key management scheme. All of this, support the idea of a similar Internet architecture for key management with some extension to handle particular constraints that present the deep space environment like disruption and long delays. It is likely to have group key management for multicast within the network.  



Symmetric key will be a requirement even for future space missions. There are sensible operations which require symmetric keys between the spacecraft and the OCC; recovery operations, low-level telemetry and telecommands, among other.  However, this is for use only within the mission and does not seem to fit for the Interplanetary Internet for scalability and interoperability issues. 



There is always a trade-off of usability against security. Security adds overhead, but the limited resources of nodes in the space segment requires the minimisation of overhead in bandwidth and storage otherwise mission planners will reject the security system \cite{book2012architecture}. 


\subsection{Delay and Disrupted Tolerant Networks}

The delay Tolerant Architecture is proposed in \cite{cerf2007delay}. It defines a network architecture for irregularly connected networks subject to frequent partition and possibly long propagation delays. To face these particular properties, the authors define an overlay layer which provides end-to-end reliable messages delivery called the \textbf{bundle layer}. This layer works under the application layer and could be used over the transport, network or data link layers using different convergence layer for each case to communicate with lower layer protocols. 

The DTN architecture provides persistent storage, hop-by-hop transfer, late binding, optional end-to-end acknowledgement to offer reliable delivery of messages. It also uses URI identifiers for naming; this flexible model allows the encapsulation of different addressing schemes. The Bundle protocol specification \cite{rfc5050}, produced by the IETF,  define the services offered by the Bundle protocol, bundle format, bundle processing and convergence layers. A node implementing the bundle protocol is called bundle node. 

 




\subsection{Security in space DTN}


 From the beginning, the Delay-Tolerant Research Group worked on the security specifications for the Bundle protocol. The motivation was to provide data integrity and confidentiality services to the bundle layer. The Bundle Security Protocol Specification \cite{rfc6257} partially defines the security blocks, security processing, allowed ciphersuites, tunnel encapsulations, policies and key management. The Bundle Protocol security Specification \cite{ietf-dtn-bpsec-07} extends the previous document adding the design decisions, canonical form of blocks, security considerations, adversary model, and more.  There are many other documents related to security which complement these specifications.  It is worth to mention that many specifications are still \textit{work in progress} and there are issues that remain unsolved.
 
 A space DTN operates as an overlay network on the top of different lower layers. Thus, the first threat are non-DTN nodes which could exploit vulnerabilities in the bundle layer. Besides, the Bundle protocol allows a mix of security-aware nodes and non-secure nodes in the same network; in a space scenario, this could be for the lack of node resources or mission policies, for instance, security being implemented in other layers. Unauthorised access is a major problem for a space DTN because the resources are scarce, especially in the space segment. Denial of service (DoS) attack is another concern for a space DTN. Attackers could take advantage of long delays to make DoS attacks more effective. 
 
 
 
 



