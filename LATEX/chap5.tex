
\section{Discussion and Challenges}
\label{sec:evaluation}

Understanding the future operation of the Interplanetary Internet is critical to choose the best key management system and the trust model. Namely, the infrastructure available, the routing model, and the trust model. Comments from professionals from the space industry were constructive to review these topics.


\subsection{Implementation Aspects}

The first point to discuss is the infrastructure available in deep space. The question is whether an ``interplanetary backbone'' or a peer-to-peer style will be used for routing. For the interplanetary backbone approach, DTN nodes have to wait for a contact opportunity with a relay satellite to send bundles. In the peer-to-peer style, nodes use other nodes to route bundles until the destination. In both cases, the contact planning authority determines the routing policies \cite{ietf-dtn-bpsec-07,cerf2007delay}. DTN nodes could receive the policies via a central authority or use \textit{ad-hoc} protocols. 

Choosing the public key management model is not a trivial task. This problem is arguably one the most complex on terrestrial networks, and the Interplanetary Internet presents the same issue. There are two possible models: single authority and multiple authority. 

In the single model, space agencies agree to trust in a unique entity like a Certificate Authority CA. A reasonable trust relationship is required between space agencies if they are meant to share infrastructure in space. Ivancic \cite{ivancic2009security} mentions that the Interplanetary Internet could look like a closed network open only to trusted parties. 

The second option is a multiple-authority model; for instance, each space agency or group could administrate one CA, creating a connected certification model. There are many techniques to deal with a connected model, and the problems involved are well-known and documented \cite{al2003certificateless,adams2004pki,gutmann2002pki}. 

According to the belief of experts in the space industry (NASA), the answer to all these issues is the same: All options will have their place in the Interplanetary Internet. The network topology will be a bunch of disjoint networks rather than a single Solar System Internet to the disadvantages of all participants. Some space agencies will cooperate and form a large space infrastructure, but others will be reluctant to cooperate, just as their corresponding nation. 

Regarding routing, the backbone model and the \textit{ad-hoc} model will have their place. The interplanetary backbone is bound to emerge at some point in the future, but there will always be situations where DTN nodes will send bundle across the network using other peers. For some parts of the network, the routing information will come from a central authority, but at the edges of the interplanetary network, the contact planning will be local. 

It is very likely that some of these networks will use the single authority approach; meanwhile, others will use the multiple authority model. For some application domains, key escrow may not be a problem, and a solution based on ID-PKC could fit, but others will require more independence from the authority. The initialisation and distribution mechanism will change according to the application domain.


Even if stakeholders consider this future for the Interplanetary Internet, this does not mean that the scenario could change eventually. In any case, a significant number of space agencies are already cooperating, for example in the International Space Station.  The single authority model is difficult to achieve, but space exploration has shown that collaboration between nations is possible and this could motivate the single CA approach. This outcome of this issue is rather political than technical, and some of these aspects are discussed below. 


\subsection{Astropolitics: Politics of Outer Space}

Currently, the idea of a single Internet Solar System is far away. Political elements play a particular role in space exploration missions, sometimes to the benefits of the participants, others to the disadvantage. For instance, as a reaction to the launch of the Sputnik from the U.S. citizens, the U.S. government created the NASA and ARPA agencies to compete against the Soviet Union in the Space Race.

% Many year later, a project on DARPA (previously ARPA) developed and tested ARPANET, the first wride-are packet-switched network, the predeceasing of  TCP/IP protocol suite.


The Outer Space Treaty \cite{ireland1967treaty} states that space exploration and use of outer space shall benefit all people and nations, which is not the case in practice. Currently, U.S. legislation prevents any kind of NASA cooperation with China, including the International Space Station (ISS). The U.S. government argues that the intentions of China for space exploration are not clear and they might have military objectives. 

The U.S. International Traffic in Arms Regulation (ITAR) presents another problem. This regulatory regime restricts and controls the export of defence and military technology to safeguard national security. Ivancic \cite{ivancic2009security} states that ITAR might present a barrier for a shared space network.

Historically, space exploration came to appease tension between nations. The flight Apollo-Soyuz was the first joint mission between the United States and the Soviet Union during a tense moment in the cold war.  This mission set a precedent of cooperation that continued with the Shuttle-Mir Program \cite{moltz2011politics}. 

In 1995, Bill Clinton declared that the tactical nuclear missiles of the United States is not pointing Russia anymore and vice-versa, and announced the cooperation with Russia and other countries to create ISS. 

The role of the ISS is scientific, commercial, educational and diplomatic \cite{costantini2014international}. Astronauts and visitors from 18 nations visited the ISS by the end of 2017. These are clear precedents of the use of space exploration in geopolitical scenarios. 


Cooperation is beneficial for both side, China is relatively new in space exploration, and experiences for past space missions could avoid problems like the Tiangong-1, which lost communication with its mission control centre and later re-enter into Earth atmosphere without control. At the same time, NASA show interest in advance of the Chinese space exploration program and the cost of space missions could be reduced dramatically. 

Nowadays, the situation is far from being solved, but it is clear that both nations need to give some ground to reach an agreement. The experience has shown that space exploration could improve the relationship between countries. China has been cooperating with other organisations like the European Space Agencies (ESA) and Roscosmos from Russia. The United States has a long history of cooperation with these agencies. 

A first step could be a mission similar to the Apollo-Soyuz, even if there is no a directly scientific benefit from it. Organisations like ESA, Roscosmos, and the United Nations could act as guarantor between the two countries to the benefit of all. The reaction from the public could help to take this cooperation further away.

\subsection{Scalability of Interplanetary Networks}

The access to authenticated information, e.g. certificates, has been identified in this project as the fundamental problem for key management. In consequence, REQ 1 states that certificates must be store ``locally''. The common solution is to store the information in the same physical node. This generate the problem of store the corresponding public keys of all other bundle nodes in one application domain. 

One could argue that certificates are not significant large and the number of spacecraft will not increase dramatically. The number of human-made objects sent into the space so far is approximately 16000; the number of satellites orbiting the planets were 4635 in 2017, an almost 10 \% increase from 2016. On the top of this, one public certificates correspond to one bundle node, one spacecraft could contain many bundle nodes as stated before. Furthermore, routing and policy information need to be stored.      

To overcome this situation, some sort of directory service could be deployed in-space for space nodes with limited resources that are not keen on tolerating the storage overhead. Dedicates satellites will offer this directory service among with other services, similarly to relay satellites. An interplanetary network will contain heterogeneous nodes, and the solution must contemplate all kinds of nodes. 


\subsection{Design Decision}


Templin \cite{templin-dtnskmreq-00} sketches 9 requirements and 4 design criteria for a delay-tolerant key management system, analysed in section \ref{sec:taxonomy}. The requirements not considered in this project are multiple points of authority and no veto capabilities, both requirements refer to the same idea: level of trust. These requirements fit only for a subset of interplanetary networks; in consequence, are not considered mandatory. 

Cryptographic systems based on symmetric keys may be a requirement even for future space missions. There are sensitive operations which require symmetric keys between the spacecraft and the mission control centre. Examples include recovery operations, low-level telemetry and low-level telecommands. However, there are some ideas on the DTN working groups involving the use of applications on the top of the bundle layer and class of services to handle those operations. 

\subsubsection{Certificates and Identity Based Cryptography}

The solutions studied in this work are mainly based on public certificates and Identity Based Cryptography (IBC). Farrell \textit{et al.} \cite{irtf-dtnrg-sec-overview-06} state that IBC only solves the problem superficially because checking the public parameters is equivalent to verify public keys. In contrast,  Asokan \cite{asokan2007towards} argue that these parameters are long-lived and comparable to a root public keys. 

In the same work, the author presents a private message passing system using IBC and traditional PKC. The result shows that IBC improves efficiency for confidentiality services but not for authentication. Therefore there is no advantage in practice in Asokan system \cite{asokan2007towards}.  

Identity-based cryptography presents other problems such as key escrow and revocation. The solution for revocation adopted in many works is time-based key, but the cost of generating and (especially) distributing new keys for short time periods in interplanetary networks is prohibitive. It is suggested that the use of ID-PKC is restricted to closed groups or applications with limited security requirements \cite{al2003certificateless}, which may be the case in some particular application domains, like an interplanetary network administrated by one organisation, but not on a large scale. Within the DTN working group, the common consensus is that ID-PKC does not provide a substantial benefit, so public certificates are preferred. 

\subsubsection{Practical Security}

One lesson from the Streamline Bundle Security Protocol (SBSP) is that protocols, policies, and configuration should have a clear separation. This provides simplicity to understand the behaviour of the protocol and flexibility to support different deployments.  The SBSP could be implemented in different types of mission, and the configurations and policies will vary depending on the mission profile and infrastructure available.  

The BSP define blocks, processing, ciphersuites, configuration and complex security goals, making implementation difficult. Mission planners (for some mission) may be reluctant to implement mandatory ciphersuites or heavy registration mechanism like the one in the DTKA specification. Then, it is desirable to separate protocols, policies, and configurations.    

Finally, "The Interplanetary Internet named was deliberately coined to suggest a far-future integration of space and terrestrial communication infrastructure to support the migration of human intelligence through the Solar System" \cite{burleigh2003interplanetary}. The best scenario for the Interplanetary Internet will be implementing a key management system similar to the Internet (TLS certificates), in the absence of a solid justification.

\subsection{Next steps}

In the same way that the Internet has revolutionised terrestrial communication, the Interplanetary Internet could produce a similar expansion in space exploration. The Bundle Protocol provided the first big step but there is still much work to be done.

The developing and testing of application for delay tolerant environments is a primary task. The bundle protocol provides the infrastructure, but the applications have to adapt to the environment.  So far, there is no much work application and security, but the IETF DTN working group is developing a mechanism for administration of asynchronously applications \cite{birrane-dtn-adm-agent-04}, but much work is needed in this area. 

Another pending topic is the standardisation of policies, configuration, and best practices as soon as possible. The obvious candidate for this work is the CCSDS. There is a vast amount of documentation for current space missions, but this scenario will not be applicable to future space missions using interplanetary shared networks.




