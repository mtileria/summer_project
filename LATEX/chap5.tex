
\section{Discussion and Challenges}
\label{sec:evaluation}

Understanding the future operation of the Interplanetary Internet is critical to choose the best key management system and the trust model. Namely, the infrastructure available, the routing model, and the trust model. Comments from professionals in the space industry were constructive to review these topics.


\subsection{Implementation aspects of the Interplanetary Internet}

The first point to discuss is the infrastructure available in deep space. The question is about whether an ``interplanetary backbone'' will be available or the routing will be performed in a peer-to-peer fashion. For the interplanetary backbone approach, DTN nodes have to wait for a contact opportunity with a relay satellite to send bundles. In the peer-to-peer style, nodes use other nodes to route bundles until the destination. In both cases, the contact planning determines the routing policies \cite{ietf-dtn-bpsec-07,cerf2007delay}. 

Some sort of contact planning authority is required to generate and distribute routing updates. The information has to be authenticated, just like certificates in PKC. Some researchers suggest that the distribution of routing updates overlaps in some extent with key distribution (find cites). 

Choosing the public key management model is not a trivial task. This problem is arguably one the most complex on terrestrial networks, and the Interplanetary Internet presents the same issue. Space agencies could agree to trust a single entity like a Certificate Authority CA.  The second option is a multiple-authority model; for instance, many space agencies could administrate the CA, or each space agency could administrate one CA, and this results in a connected certification model.

The single authority model is indisputable hard to achieve, but space exploration has shown that collaboration between nations is possible and this could motivate the single CA approach. A natural question is whether space agencies can continue this cooperation to simplify the key management or not. Furthermore, a reasonable trust relationship is required between organisations which are meant to share infrastructure in space. Ivancic \cite{ivancic2009security} mention that the Interplanetary Internet could look like a closed network open only to trusted parties. 

The second option requires some kind of cross-organisational certification which carries well-known problems in the broad adoption of public key certificates. In both cases, the role of the private sector in the space industry presents new complications. It is expected that in the next ten years, the private sector will play a major role in space exploration and space-based networks will tend to be more open. 


According to the expert in the space industry, the answer to all these topics is the same: All options will be realised in the Interplanetary Internet. The network topology will be a bunch of disjoint networks rather than a single Solar System Internet to the disadvantages of all participants. Some space agencies will cooperate and form a more substantial space infrastructure, and others will be reluctant to cooperate, just as their corresponding nation. In particular situations, one space agency could participate in a joint network with other agencies but remain some space assets for internal use.  

Regarding routing, the belief is that the backbone model and the \textit{ad-hoc} model will have their place in the Interplanetary Internet. The interplanetary backbone is bound to emerge at some point in the future, but there will always be situations where DTN nodes will send bundle across the network using other peers. For some parts of the network, the routing information will come from a central authority, but at the edges of the interplanetary network, the contact planning will be local. The certificate model presents the situation, some application domains will use the multiple-authority model, but others will use a single CA. In the latter case, a much simpler initialisation and distribution mechanism are required. 

\subsection{Design decision}


Templin \cite{templin-dtnskmreq-00} sketches 9 requirements and 4 design criteria for a delay-tolerant key management system. The design criteria list is defined to enforce the key management system requirements. The requirements not considered in this project are multiple points of authority and no veto capabilities, both requirements refer to the same idea, level of trust. The idea is that DTN nodes must not be forced to trust in a single authority and no single compromised authority can spoil the system. Considering the implementation aspects of the Interplanetary Internet discussed before, these requirements fit only for a subset of interplanetary networks. In consequence, those cannot be regarded as a fundamental requirement. 

Cryptographic systems based on symmetric keys may be a requirement even for future space missions. There are sensitive operations which require symmetric keys between the spacecraft and the mission control centre. Examples include recovery operations, low-level telemetry and low-level telecommands. However, there are some ideas on the DTN working groups involving the use of applications on the top of the bundle layer and class of services to handle those operations. 

%There are decentralised key management systems without a trusted party like Web of Trust or block-chain based. The problem is that revocation of public keys becomes impossible or unsuitable in a real scenario. The need of a trusted party acting as Certificate Authority becomes a requirement. Another problem is the message overhead in these model, something that it is unlikely to be acceptable in any king of delay tolerant network. 

   

The key management systems study in this work are mainly based on public certificates and Identity Based Cryptography (IBC). Farrell \textit{et al.} \cite{irtf-dtnrg-sec-overview-06} state that IBC only solves the problem superficially because checking the public parameters is equivalent to verify public keys. In contrast,  Asokan \cite{asokan2007towards} argue that these parameters are long-lived and comparable to a root public keys. In the same work, Asokan presents a private message passing system using IBC and traditional PKC. The result is that IBC improves efficiency for confidentiality services but not for authentication, therefore there is no advantage in practice.  

Identity-based cryptography presents other problems such as key escrow and revocation. The solution for revocation adopted in many works is time-based key, but the cost of generating and (especially) distributing new keys for short time periods in interplanetary networks is prohibitive. These problems suggest that the use of ID-PKC is restricted to closed groups or applications with limited security requirements \cite{al2003certificateless}, which may be the case in same particular deployments of the Interplanetary Internet, like an interplanetary network administrated by one organisation, but not on a large scale.

\subsection{A perspective of the future}

As the internet has revolutionized terrestrial communication allowing diverse reigional networks to confederate over a backbone, the emergence of conceot indicates a similar expansion into space. 

Finally, "The Interplanetary Internet named was deliberately coined to suggest a far-future integration of space and terrestrial communication infrastructure to support the migration of human intelligence through the Solar System" \cite{burleigh2003interplanetary}. The best scenario for the Interplanetary Internet will be implementing a key management system similar to the Internet (TLS certificates), especially if the new scheme does not present substantial benefits. 

